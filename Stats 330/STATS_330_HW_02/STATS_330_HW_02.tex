
\documentclass[11pt]{article}
\usepackage{enumerate}
\usepackage{amsfonts}
\usepackage{amsmath}
\usepackage{mathabx}
\usepackage{graphicx}
\usepackage{adjustbox}
\usepackage{parskip}


\begin{document}

\title{Stats 330 Assignment 2}
\author{Adam Hammes $\bullet$ hammesa@iastate.edu $\bullet$ Section B}
\maketitle

\section*{Problem 1}
	\begin{enumerate}[(a)]
		\item Ordered drawing without replacement. $P( 7, 3) = \dfrac{7!}{4!} = 
		210$.
		
		\item Ordered drawing with replacement. $7^3 = 343$.
		
		\item Unordered (note the key word grouping) drawing without replacement. $\binom{7}{3} = \dfrac{7!}{4!3!} = 35$.
		
		\item We have $n= 7$ objects and need to choose $r=3$ of them. Using the formula for combinations with replacement, we get
		\begin{align*}
		\binom{n+r-1}{r} = \binom{7+3-1}{3} = \binom{9}{3} =\dfrac{9!}{6!3!} = 84 
		\end{align*}
		
	
	\end{enumerate}
	
\section*{Problem 2}

From the 7 remaining digits, we have to pick a sequence of 4 distinct digits, which is a permutation. $P(7, 4) = 7!/3! = 840$. Ignoring the restrictions, the number of possible permutations is $7^4$. Our probability is thus $\dfrac{840}{7^4} \approx 0.35$.

\section*{Problem 3}

The formula for probability of equally likely, independent events is $\dfrac{|E|}{|\Omega|}$. We must find the size of both sets.
\begin{itemize}
	\item $\Omega$ is the total number of possible license plates. For the letters, there are 26 possibilities each for 3 spots, and then 10 possibilities for 3 spots for the numbers. $|\Omega| = 26^3 * 10^ 3 = 1.7576 * 10^7$.
	
	\item $E$ is similar to $\Omega$, but with less possible letters and numbers. $|E| = (26-3)^3 * (10-2)^3 = 6229504$.
\end{itemize}	

Using these numbers in the formula, we find that the probability is
	\begin{align*}
		\dfrac{|E|}{|\Omega|} = \dfrac{6229504}{1.7576 * 10^7} \approx 0 .354
	\end{align*}


\section*{Problem 4}
	$\Omega$ is the number of 4-combinations possible from a set of 12. Therefore $|\Omega| = \binom{12}{4} = \dfrac{12!}{4!\ 8!} = 495$. For the two problems we simply need to find the size of the event and divide by 495.
	
\begin{enumerate}[(a)]
	\item Because there are 4 ethnic groups and only four spots, we simply need to choose the representative from each ethnicity. $\binom{n}{1} = n$, so $|E| = 5*3*2*2 = 60$ and the probability is $60/495 = 4/33$.
	\item  First find the number of committees with no caucasians, $|E'|$. To pick this group we simply select four members from the 7 non-caucasian members, $\binom{7}{4} = \dfrac{7}{4!3!} = 35$. $P(E) = 1- P(E') = 1 - \dfrac{35}{495} = \dfrac{92}{99}$.
\end{enumerate}

\section*{Problem 5}
Two of the cards are aces. From the remaining 50 cards, we need to pick a combination of three cards to flesh out the hand.
\begin{align*}
	\binom{50}{3} = \dfrac{50!}{47!3!} =19,600
\end{align*}

The number of possible 5-card hands is $\binom{52}{5}$, so the probability becomes $\dfrac{19,600}{\binom{52}{5}} = \dfrac{5}{663}$. 

\section*{Problem 6}
Assume that the spyware chooses passwords to try randomly. Also, let $G$ denote the probability of the spyware guessing the password.

\begin{enumerate}[(a)]
	\item $G = \dfrac{10^6}{P(26, 6)} = \dfrac{10^7}{26!/20!} \approx 0.00603$
	\item $G = \dfrac{10^6}{P(52, 6)} = \dfrac{10^7}{52!/46!} =$\\
	$\dfrac{1}{4580667}$
	\item $G = \dfrac{10^6}{52^6}$ I'm not going to attempt to calculate this.
	\item Assuming case sensitive, $G = \dfrac{10^6}{62^6}$; else $G = \dfrac{10^6}{36^6}$

	
\end{enumerate}

\end{document}