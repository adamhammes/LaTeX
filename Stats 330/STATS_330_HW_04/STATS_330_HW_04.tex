
\documentclass[11pt]{article}
\usepackage{enumerate}
\usepackage{amsfonts}
\usepackage{amsmath}
\usepackage{mathabx}
\usepackage{graphicx}
\usepackage{adjustbox}
\usepackage{parskip}


\begin{document}

\title{Stats 330 Assignment 4}
\author{Adam Hammes $\bullet$ hammesa@iastate.edu $\bullet$ Section B}
\maketitle

\section*{Problem 1}
\begin{enumerate}[(a)]
	\item
	\begin{tabular}{l| c c c c|}
		$x$ & 0 & 1 & 2 & 3 \\
		\hline
		$p_{\chi}(x)$ & 0.5 & 0.25 & 0.1 & \textbf{0.15}\\
	\end{tabular}

	\item $P ( x \ge 2) = 1 - (0.5 + 0.25 + 0.1 ) = 0.15$
	
	\item $P ( x \neq 0 \wedge x \neq 2 ) = 1 - (0.5 + 0.1 ) = 0.4$
	
	\item $P ( x \ge 0 ) = 1$
	
	\item
	\begin{enumerate}[i.]
		\item $E[X] = 0 * 0.5 + 1 * .25 + 2 * .1 + 3 * .15 = 0.9$
		
		\item $V[X] = 0.5(0 - 0.9)^2 + .25(1-.9)^2+ .1(2-.9)^2  + .15(3-.9)^2 = 1.19$
	\end{enumerate}
\end{enumerate}


\section*{Problem 2}
\begin{enumerate}[(a)]
	\item $Im(Y) = \{-1, 1, 3, 5\}$
	
	\item We can simply apply the transformation we used to find $Y$ to calculate $E[X]$ (properties of expectations, p. 49 in text):
		\begin{align*}
		E[Y] &= 5 - 2( E[X] ) \\
		&= 5 - 2( 0.9 ) \\
		&= 3.2
		\end{align*}
		
		As for $Var[Y]$, the relevant formula is (p. 52 of text):
		\begin{align*}
			Var(aX + b) &= a^2\ Var[X] \\
			Var[Y] &= 2^2\ (1.19) \\
			&= 4.76
		\end{align*}
		

\end{enumerate}



\end{document}