\documentclass[11pt]{article}
\usepackage{enumerate}
\usepackage{amsfonts}
\usepackage{amsmath}
\usepackage{mathabx}
\usepackage{graphicx}
\usepackage{adjustbox}
\usepackage{parskip}


\begin{document}

\title{Stats 330 Assignment 4}
\author{Adam Hammes $\bullet$ hammesa@iastate.edu $\bullet$ Section B}
\maketitle

\section*{Problem 1}
\begin{enumerate}[(a)]
	\item
	\begin{tabular}{l| c c c c|}
		$x$ & 0 & 1 & 2 & 3 \\
		\hline
		$p_{\chi}(x)$ & 0.5 & 0.25 & 0.1 & \textbf{0.15}\\
	\end{tabular}
	
	\item
	\begin{tabular}{l| c c c c|}
		$x$ & 0 & 1 & 2 & 3\\
		\hline
		$P\{X \leq x\}$ & 0.5 & .75 &.85 & 1\\
	\end{tabular}
	\item
	\begin{enumerate}[i.]
		\item $P ( x \ge 2) = 1 - (0.5 + 0.25 + 0.1 ) = 0.15$
	
		\item $P ( x \neq 0 \wedge x \neq 2 ) = 1 - (0.5 + 0.1 ) = 0.4$
	
		\item $P ( x \ge 0 ) = 1$
	\end{enumerate}
	\item	
	$E[X] = 0 * 0.5 + 1 * .25 + 2 * .1 + 3 * .15 = 0.9$
		
	$V[X] = 0.5(0 - 0.9)^2 + .25(1-.9)^2+ .1(2-.9)^2  + .15(3-.9)^2 = 1.19$

\end{enumerate}


\section*{Problem 2}
\begin{enumerate}[(a)]
	\item $Im(Y) = \{-1, 1, 3, 5\}$
	
	\item We can simply apply the transformation we used to find $Y$ to calculate $E[X]$ (properties of expectations, p. 49 in text):
		\begin{align*}
		E[Y] &= 5 - 2( E[X] ) \\
		&= 5 - 2( 0.9 ) \\
		&= 3.2
		\end{align*}
		
		As for $Var[Y]$, the relevant formula is (p. 52 of text):
		\begin{align*}
			Var(aX + b) &= a^2\ Var[X] \\
			Var[Y] &= 2^2\ (1.19) \\
			&= 4.76
		\end{align*}
\end{enumerate}

\section*{Problem 3}
\begin{enumerate}[(a)]
	\item
		\begin{align*}
			E[Z] &= \frac{1}{n} (n \times E[X]) \\
			&= E[X] \\
			&= 0.9 \\
		\end{align*}
	\item
		\begin{align*}
			Var[Z] &= \frac{1}{n}\ Var[n \times Var[X]] \\
			&= \frac{1}{n} \times n^2 \times Var[X] \\
			&= 1.19\ n
		\end{align*}
	
\end{enumerate}


\section*{Problem 4}
Because every die roll is equally likely, we can pull out the probability $1/6$ and multiply it through later.
	\[E[X] = \dfrac{1}{6}( 1+2+3+4+5+6) = 3.5 \]
We can also do this with variance; in addition, since the image and probabilities are completely symmetric around 3.5, we can calculate only the lower half of the values and multiply by two.
	\[Var[X] = \dfrac{1}{6} * 2 * ( (1-3.5)^2 + (2 - 3.5)^2 + (3-3.5)^2 ) = \dfrac{35}{12} \]
	
\section*{Problem 5}
\begin{enumerate}[(a)]
	\item 
	$E[\text{100 shares of A}] = -2 \times 0.5 + 2 \times 0.5 = 0$
	
	$Var[\text{100 shares of A} ] = (-2 -0)^2  \times 0.5 + (2-0)^2 \times 0.5 = 4$
	
	\item
	$E[\text{100 shares of B}] = 4 \times 0.2 + -1 \times 0.8 = 0$
	
	$Var[\text{100 shares of B}] = (4-0)^2 \times 0.2 + (-1 - 0)^2 \times 0.8 = 4$
	
	\item
	The expected value of each function is 0, so mixing our strategies just gives us $E[\text{mixed strategy}] = 0$
	
	The formula for independent $X$ and $Y$ is $Var(X+Y) = Var(X) +Var(Y)$; since the two strategies each account for half of the total shares, $Var[\text{mixed strategy}] = 4$.

\end{enumerate}


\section*{Problem 6}
We can model the search engine crawling web sites as individual Bernoulli experiments with $\rho = 0.2 = $ probability that the site contains the keyword.

\begin{itemize}
	\item
	Since we are conducting of searches (Bernoulli experiments), we can model this problem as a Bernoulli distribution. 
	\[ P_x(5) = 0.2^5\ (1-0.2)^5\ \binom{10}{5}  \approx 0.026\]
	
	\item 
	We are concerned with the number of Bernoulli experiments required before success; a geometric distribution is natural.
	\[P(x \ge 5) = 1- F_x(4) = 1 -(1- (1-0.2)^4) = 0.4096 \]
	
\end{itemize}
\end{document}