\documentclass[11pt]{article}
\usepackage{enumerate}
\usepackage{amsfonts}
\usepackage{amsmath}
\usepackage{mathabx}

\begin{document}

\author{Adam Hammes}
\title{Stats 330 Homework 1}
\maketitle

\section*{Problem 1}
	\begin{enumerate}[(a)]
	\item
		Let $hth$ denote the results of flipping a coin three times and getting a heads, then a tails, then a heads in that order. $\Omega$ then consists of the following:	
		\begin{itemize}
			\item $hhh$
			\item $hht$
			\item $hth$
			\item $htt$
			\item $thh$
			\item $tht$
			\item $tth$
			\item $ttt$
		\end{itemize}
	
	\item 
		\begin{enumerate}[i.]
			\item $A$ = $\{ hhh,\ hht,\ hth,\ thh\}$
			\item $B$ = $\{ hhh,\ hht \}$
			\item $C$ = $\{ hht,\ htt,\ tht,\ ttt\}$
		\end{enumerate}
		
	\item
		\begin{enumerate}[i.]
			\item $\overline{A} = \{htt,\ tht,\ tth,\ ttt\} $
			\item $ A \cap C = \{ hht \} $
			\item $ A \cup C = \{ hhh,\ hht,\ hth,\ thh,\ htt,\ tht,\ ttt\}$

		\end{enumerate}
	\end{enumerate}

\section*{Problem 2} 
	\begin{enumerate}[(a)]
		\item
		\begin{itemize}
			\item $ccc$
			\item $ccs$
			\item $csc$
			\item $css$
			\item $scc$
			\item $scs$
			\item $ssc$
			\item $sss$		
		\end{itemize}	
		
		\item
			Since all outcomes are equally likely, and only one of the eight possible 
			outcomes results in the commuter not stopping, the probability is $\dfrac{1}
			{8} = 0.125$.
			
		\item
			\begin{enumerate}[i.]
			\item $ A = \{ scc,\ scs,\ ssc,\ sss\}$
			\item $ B = \{ csc,\ css,\ ssc,\ sss\}$
			\item $ \overline{B} = \{ ccc,\ ccs,\ scc,\ scs \}$
			\item $ A \cup B = \{ scc,\ scs,\ ssc,\ sss,\ csc,\ css \}$
			\item $ A \cap B = \{ ssc,\ sss \}$
			\item $ A \cap \overline{B} = \{ scc,\ scs \}$
			\end{enumerate}
	\end{enumerate}
	
\section*{Problem 3}
	\begin{enumerate}[(a)]
		\item  $\Omega =$	\{(1,1),	(1,2),	(1,3), (1,4),	(1,5),	(1,6),
		
					(2,1),	(2,2),	(2,3),	(2,4),	(2,5),	(2,6),

					(3,1),	(3,2),	(3,3),	(3,4),	(3,5),	(3,6),

					(4,1),	(4,2),	(4,3),	(4,4),	(4,5),	(4,6),

					(5,1),	(5,2),	(5,3),	(5,4),	(5,5),	(5,6),

					(6,1),	(6,2),	(6,3),	(6,4),	(6,5),	(6,6)\}
					
		\item 0
		\item $\dfrac{5}{36}$
		\item $\dfrac{7}{36}$

	\end{enumerate}
	
\section*{Problem 4}
	Let the good computer chips be numbered $1,\ 2,\ 3,\ 4$, and the bad ones $a,\ b$. Then $\Omega = \{ $ 
	
	(1,2),  (1,3), (1,4), (1, a), (1, b),
	
	(2, 3), (2, 4), (2, a), (2, b),
	
	(3, 4), (3, a ), ( 3, b), (4, a), (4, b), (a, b) \}\\\\
	Since there is only one way of drawing two bad chips out of the bag, the probability of such an event is $\dfrac{1}{|\Omega|} = \dfrac{1}{15}$.
	
	
\section*{Problem 5}
	From the book we have $P( MB ) = .4$, $P( HD) = .3$, and $ P( MB \cap HD ) = .15$. Add the probabilities $P( MB )$ and $ P( HD)$, then subtract $P( MB ) \cap P(HD)$ to avoid double-counting. This will give us the probability that there \textit{is} a failure. To find the probability that there \textit{isn't} a failure, we subtract the previous probability from 1. 
	
	\[1- (.4 + .3 -.15) =  .45\]
	
\section*{Problem 6}
	\begin{enumerate}[(a)]
		\item $P( \c{F} ) = 1 - P( F ) = 1-.6 = .4$
		\item \begin{align*}
			P( \overline{F} \cap \overline{C} ) &= 1- (\ P( F) + P( C ) - P(F \cap C\ ) \\
			&= 1-(.7 + .6 - .5)\\
			&= .2
		\end{align*}
		
		\item
		\begin{align*}
		P( C - F ) &= P(C) - P ( C \cup F )\\
		&= .7-.5\\
		&= .2
		\end{align*}
		
		\item
		\hfill \begin{align*}
		P( F - C ) &= P(F) - P( C \cup F)\\
		&= .6 -.5\\
		&= .1
		\end{align*}

	\end{enumerate}

	
	  
	  
	
	
	
	
	
\end{document}