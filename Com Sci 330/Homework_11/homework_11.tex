\documentclass[11pt]{article}
\usepackage{enumerate}
\usepackage{amsfonts}
\usepackage{amsmath}
\usepackage{mathabx}

% makes quotation marks show up correctly
\usepackage [english]{babel}
\usepackage [autostyle, english = american]{csquotes}
\MakeOuterQuote{"}

\begin{document}

\let\iff\leftrightarrow

\title{Com Sci 330 Assignment 11}
\author{Adam Hammes, hammesa@iastaste.edu}
\maketitle

\section*{Problem 1}
	Let $n$ be the number of people to choose from. Choosing a vice-president and president and then choosing a committee of size $k-2$ from the 
	remaining people can be expressed as $P(n,\ 2)\times C(n-2,\ k-2)$. This is the same as first picking all the people who are either on the 
	committee, president, or vice-president, and then choosing the president and vice-president, written as $C(n,\ k)\times P(k,\ 2)$. Since this 
	same concept is expressed equally well either way, the two formulas are equivalent.


\section*{Problem 2}
	\begin{enumerate}[(a)]
	\item
		Consider a population of $2n$ people, evenly split between and women. A handshake requires exactly two people, so the total number of handshakes possible is $C(2n,\ 2)$.
		
		Another way to look at it is to calculate the same-sex handshakes. Since there are $n$ men and $n$ women, the number of possible same-sex handshakes is $2 \times C(n,\ 2)$. To calculate the inter-gender handshakes, we first pick a man, $n$ possibilities, and then a women, which combines to give $n^2$ possibilities. Therefore the total number of handshakes possible is $2\times C(n,\ 2) + n^2$.
		
		Since both formulas count the same concept, they are equivalent.
	\item
		\begin{align*}
		\binom{2n}{2} &= \dfrac{(2n)!}{2(2n-2)!}\\
		&= \dfrac{(2(n))!}{2(2(n-1)!)}\\
		&= \dfrac{1}{2} \times 2n(2n-1)\\
		&= n(2n-1)\\
		&= n^2 -n + n^2\\
		&= n(n-1) + n^2\\
		&= 2\times \dfrac{n!}{2(n-2)!} + n^2\\
		&= 2\binom{n}{2} + n^2 && \text{ QED.}
		\end{align*}

	\end{enumerate}


\section*{Problem 3}
	Consider a population of $n$ men and $n$ women. To choose a group of size $r$, you can simply calculate $\binom{2n}{r}$. 
	\\\\
	A second way to calculate this is to iterate through all possible numbers of men in the group. Since a person is either a man or 
	woman, and their are $r$ people in the group, if there are $k$ men then there are $r-k$ women. Therefore we can write the number of 
	combinations as the sum of picking $k$ men from the men half of the population and then $r-k$ women from the women half, for $0 \le k \le r$. For 
	example, when $k= 0$, we are calculating the combinations when there are 0 men and $r$, and when $k=1$, we are calculating the combinations where 
	there is exactly 1 guy and $r-1$ women. Summing these "sub"-combinations gives us the total number of combinations. Put formally, this is:
	
	\begin{align*}
	\sum\limits_{k=0}^r \lbrack C(n,\ k) \times C(n,\ r-k) \rbrack
	\end{align*}\\
	Since they both express the same concept, the two formulas are equivalent.

\section*{Problem 4}
	\begin{enumerate}[(a)]
	\item
		One way to think of this problem is with cookies and lines. Pretend we have all 24 cookies we purchase sorted by type and lined up straight 
		in a row. We can place 4 lines to separate the 5 types of cookies, for a total of $(24+4) = 28$ objects lined up sequentially. Note that we 
		only need four lines- one line to demarcate the end of each type of cookie, and no line needed at the end of the fifth type since it's the 
		end of the cookie line. To represent each possible combination we simply choose which 4 of those 28 objects are the lines.
		
		\begin{align*}
		\binom{28}{4} = 20,475 \text{ possible ways.}
		\end{align*}
	
	\item
		There is no change to this problem, except that $5 \times 3 = 15$ cookies are already accounted for. Therefore we can repeat the 
		calculation in (a) with a sequence of $28-15 =13$ objects.
		
		\begin{align*}
		\binom{13}{4} = 715 \text{ possible ways.}
		\end{align*}
		
	\item
		Since five chocolate chip cookies and three oatmeal cookies have been chosen, we only need to buy $24-5-3= 16$ cookies. Additionally, since 
		we know we can't buy anymore oatmeal cookies, we really only have four types to choose from. Using the formula from (a), we get:
		
		\begin{align*}
		\binom{16+4-1}{3} = 969 \text{ possible ways.}
		\end{align*}
		
	\item
		To solve this problem, we will calculate the number of possibilities ignoring the sugar cookies stipulation, then subtract all 
		possibilities that contain at least five sugar cookies.
		
		For the first part, use the formula from (a) and pick $24-5 = 19$ cookies.
		
		\begin{align*}
		\binom{19+5-1}{4} = 8,855 \text{ ways, ignoring the sugar cookies stipulation.}
		\end{align*}
		
		Now we find the number of possibilities with both at least five oatmeal cookies and five sugar cookies, and subtract that from the number 
		we arrived at earlier.
		
		\begin{align*}
		8,855 - \binom{(24-10)+5-1}{4} = 5,795 \text{ possible ways.}
		\end{align*}
	\end{enumerate}
	
\section*{Problem 5}
	Since the notebooks are identical, but the students (presumably) aren't, we can use a similar formula to that of Problem 5. Instead of laying 
	out the cookies, we lay out the ten books in a row, with five "lines" separating each student's books. We then simply have to choose which of the 
	spots the five "lines" go in. Using the previous formula, we arrive at:

	\begin{align*}
	\binom{10+6-1}{5} = 3003 \text{ possible ways.}
	\end{align*}


\section*{Problem 6}
	First, consider the number of ways to distribute the fruits among my friends without accounting for the indistinguishable objects, which is 
	simply $16!$. From this number, we have to divide out the number of (non-unique) permutations of the bananas, oranges, and apples, which are 
	$5!, \ 3!$ and $8!$ respectively. 
	\begin{align*}	
	\dfrac{16!}{5! \times 3! \times 8!} = 720,720 \text{ different possibilities.}
	\end{align*}
	
	
\section*{Problem 7}
	\begin{enumerate}[(a)]
		\item
		We simply need to choose three books to go into each box; note that once a book has been packed, it can't be packed in another box. Therefore 
		the total number of books to choose from decreases by three for each box.
		\begin{align*}
		\binom{18}{3} \times \binom{15}{3} \times \binom{12}{3} \times \binom{9}{3} \times \binom{6}{3} \times \binom{3}{3} = 137,225,088,000 \text{ 
		possibilities.}
		\end{align*}	
		
		\item
		Since there are six boxes, there are $6!$ possible ways that they can be arranged; however, since these arrangements are the same when the 
		boxes are identical, we have to divide by the number of ways.
		
		\begin{align*}
		137,225,088,000 \div 6! = 190,590,400 \text{ possible packings.}
		\end{align*}			
	
		\item
		Similar to (b), except instead of six boxes not mattering, three don't. Therefore we need to discard $3!$ possibilities.
		
		\begin{align*}
		137,225,088,000 \div 3! = 22,870,848,000 \text{ possible packings.}
		\end{align*}
	
	\end{enumerate}


\section*{Problem 8}
	\begin{enumerate}[(a)]
	\item
		There exists no simple formula for this type of problem, so I will simply enumerate the possibilities. To make this process somewhat 
		systematic I'll follow this rule: try to maximize the value as we go towards left side. Note that 5-0-0-0-0 is the same as 0-5-0-0-0, since 
		the boxes are identical.
		\begin{itemize}
			\item
				5-0-0-0-0
			\item
				4-1-0-0-0
			\item
				3-2-0-0-0, 3-1-1-0-0
			\item
				2-2-1-0-0, 2-1-1-1-0
			\item
				1-1-1-1-1
		\end{itemize}
		This gives a total of seven possibilities.
	
	\item
		This is a more familiar type of problem. Using the formula yet again, we get:
		\begin{align*}
		\binom{5+5-1}{4} = 126 \text{ possibilities.}
		\end{align*}
			
			
	\end{enumerate}

\section*{Problem 9}
	Consider $x_i$ to be expressing the number of "balls" placed in the $i$th "bin". Then this problem becomes simply how to distribute 24 
	identical balls among 5 unique bins.
	
	\begin{enumerate}[(a)]
	\item
		Since some digits have minimum values, we only have $(24-1-2-3-4-5) = 9$ balls to place. Using the formula for indistinguishable objects/
		distinguishable bins, we find:
		
		\begin{align*}
		\binom{9+5-1}{9} = 715 \text{ possible solutions.}
		\end{align*}
		
	\item
		Same problem as (a) but with $(24-3-3)=18$ balls to place.
		
		\begin{align*}
		\binom{18+5-1}{18} = 7,315 \text{ possible solutions.}
		\end{align*}
	
	\item
		First find the number of solutions, disregarding the limitation on $x_4$. Same as (a) and (b) with 21 balls.
		
		\begin{align*}
		\binom{21+5-1}{21} = 12,650 \text{ disregarding the } x_4 \text{ limitation}
		\end{align*}
		
		The we subtract the possible solutions where $x_1 \ge 3$ and $x_4 \ge 4$ (17 balls).
		
		\begin{align*}
		12,650 - \binom{17+5-1}{17} = 6,665 \text{ possible solutions.}
		\end{align*}
	\end{enumerate}
	
\section*{Problem 10}
	\begin{enumerate}[(a)]
	\item
		This is another indistinguishable objects/ distinguishable bins problem. Arrange the books in a line, and the place two lines to demarcate 
		the separation of the shelves.
		
		\begin{align*}
		\binom{10+3-1}{2} = 66 \text{ possible arrangements.}
		\end{align*}
	
	\item
		Same as (a), but since the books are different the order in which they are arranged matters. Each order can be described uniquely by 
		starting at the top left and reading rows until the bottom right, forming a sequence of books. This sequence can be arranged in $10!$ ways, 
		so the total number of arrangements is $66 \times 10! = 239,500,800$ possible arrangements.
	\end{enumerate}
	

\end{document}