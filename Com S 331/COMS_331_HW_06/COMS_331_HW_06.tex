\documentclass[11pt]{article}

\usepackage{enumerate}
\usepackage{amsmath}
\usepackage{amssymb}
\usepackage{parskip}

\begin{document}

\title{Com S 331 Homework 6}
\author{Adam Hammes $\bullet$ hammesa@iastate.edu}
\date{October 9, 2014}
\maketitle

\section*{Problem 41}

(a+b)( (b(a+b) + a)(a+b) )*b

\section*{Problem 42}

See Figures 42 a-c

\section*{Problem 43}

Let $M = (Q,\Sigma,\delta,s,F)$ be a DFA that decides a language A. I construct an NFA $N$ that decides $A'$.

The rough intuition for N is that it consists of two DFAs, top and bottom. N will read in a string on the bottom DFA until it nondeterministically "pretends" to read a 1 in addition to the current symbol, at which point it jumps to the top DFA and finishes reading the string.

The states in $N$ are $Q \times \{B,T\}$; each of the top and bottom automata gets a copy of the states in $M$. Our alphabet is the same. We start in the bottom DFA, so $S = (s,B)$. We only accept a string if we have nondeterministically read a 1 and jumped to the top DFA, so our accept states are $F \times \{T\}$.

Finally, we come to our transition function. $\Delta$ maps from $((q \in Q, a \in \{B,T\}) , b \in \Sigma)$ to $ \mathbb{P}(Q \times \{B,T\}))$, and is defined as the following:

\begin{itemize}
	\item If $a = B$: We are in the bottom DFA, so our options are to either read the symbol as normal or "pretend" to read a 1, jumping up to the top DFA and then the symbol. Therefore $\Delta((q, B), b) = \{ (\delta(q, b), B),\ (\ (\delta( \delta( q, 1), b)),\  T) \}$.
	
	\item If $a = T$: A simpler case. We have already placed our 1 into the string, so we must keep reading as normal. $\Delta((q, T), b) = \{ (\delta(q, b),\ T) \}$
\end{itemize}

Now consider $N$ reading a string $s$. We accept $s$ iff we end in an accepting state of A, in the top DFA; the only way we can move to the top is by pretending to read a 1 after some number of bits in $s$. Let all the bits before our imaginary 1 be referred collectively as $x$, and the bits after as $y$. With this new notation, we can write that $N$ accepts a string iff $ s = x1y \in A$ which is equivalent to $L(N) = A'$ by the definition of $A'$. $\checkmark$

\section*{Problem 44}
\begin{enumerate}[(a)]
\item
Let A be the set $\{ a^nb^m \mid n = 2m \}$, and $M = (Q, \{a,b\}, \delta, s, F)$ be a DFA. It suffices to prove that L($M$) $\neq$ A.

By the Pigeonhole principle, there exist integers $g,\ h$ such that $0 \leq g < h \leq |Q|$ and $\hat{\delta}(s, a^g) = \hat{\delta}(s, a^h)$. Let x = $a^{2g}b^g$ and y = $a^{2g}b^h$. Clearly, $x \in A$; also, since  $g \neq h$, $y \not\in A$.

However,
\begin{align*}
	\hat{\delta}(s, x) &= \hat{\delta}(s,\ a^{2g}b^g) \\
	&= \hat{\delta}(\ \hat{\delta} (s,\ a^{2g}),\ b^g ) \\	
	&= \hat{\delta}(\ \hat{\delta} (s,\ a^{2g}),\ b^h ) \\
	&= \hat{\delta}(s, y)
\end{align*}

Since $M$ ends in the same state for $x$ and $y$, it accepts both or neither of the strings. Therefore L($M$) $\neq$ A. $\checkmark$

\item

Let A be the set $\{ x \in \{ a,b,c \}^* \mid x = $ rev$(x) \}$, and B the set $ \{ a^n ba^n | n \in N \}$. I first prove that B is not regular.

Let $M = (Q, \{a,b\}, \delta, s, F)$ be a DFA. It suffices to prove that L($M$) $\neq$ B. By the Pigeonhole principle, there exist integers $l, k$ such that $0 \leq l < k \leq |Q|$ and $\hat{\delta}(s, a^l) = \hat{\delta}(s, a^k)$. Let $x$ = $a^lba^l$ and $y$ = $a^kba^l$. By definition of B, $x \in B$; since $l \neq k$, $y \not\in B$.

However,
\begin{align*}
\hat{\delta}(s, x) &= \hat{\delta}(s, a^lba^l) \\
&= \hat{\delta}(\ \hat{\delta}(s, a^l),\ ba^l) \\
&= \hat{\delta}(\ \hat{\delta}(s, a^k),\ ba^l) \\
&= \hat{\delta}(s, a^kba^l) \\
&= \hat{\delta}(s, y)
\end{align*}

Since $\hat{\delta}(s, x) = \hat{\delta}(s, y)$, $M$ accepts both or neither of the strings. Therefore L($M$) $\neq$ B and B is not regular.

The set a*ba* is obviously regular; which strings in the set are palindromes? The only way a member of a*ba* could be a palindrome is if the number of a's on both sides of the b are equal, which is the definition of the set B. Since A is the set of palindromes, $a^*ba^* \cap $ A = B. 

By closure of regular languages under $\cap$, if $a^*ba^*$ and A were regular, B is also regular. B was proven not to be regular, so either $a^*ba^*$ or A is irregular; $a^*ba^*$ is regular, so we can conclude that A must not be regular. $\checkmark$.




\end{enumerate}

\section*{Problem 45}

Counterexample: 10.

\section*{Problem 46}

Let $\alpha = 1$ and $\beta = 20$. We can then rewrite the left-hand expression as $0(\alpha \beta)^*\alpha2$; applying equation 9.14 from the text, we come up with the equivalent expression 
$0\alpha(\beta \alpha)^*2$. Substituting back in for $\alpha$ and $\beta$ we find that 0(120)*12 is equivalent to 01(201)*2. $ \checkmark $


\section*{Problem 47}

Counterexample: $\lambda$

\section*{Problem 48}

Counterexample: 0





\end{document}