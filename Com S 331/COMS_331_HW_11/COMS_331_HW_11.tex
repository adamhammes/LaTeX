\documentclass[11pt]{article}

\usepackage{enumerate}
\usepackage{amsmath}
\usepackage{amssymb}
\usepackage{parskip}
\usepackage{mathtools}

\DeclarePairedDelimiter\ceil{\lceil}{\rceil}
\DeclarePairedDelimiter\floor{\lfloor}{\rfloor}

\let\iff\leftrightarrow
\let\imp\rightarrow


\begin{document}

\title{Com S 331 Homework 11}
\author{Adam Hammes $\bullet$ hammesa@iastate.edu}
\date{December 4, 2014}
\maketitle


\section*{Problem 86} $B$ is regular by the following regular expression: 

$(00 + 11)^* 01 (0+1)^* $


\section*{Problem 88}
\begin{enumerate}[(a)]
	\item The last half of a palindrome is uniquely determined by the first half, so the number of palindromes is the number of possible strings that can form the first half. 
	For a string $x$ with even length, the first, the length of the first half is simply 
		\[\frac{|x|}{2} \] 
	and for odd $|x|$, the first half length is
		\[\frac{|x+1|}{2} \]
	since the middle bit is included in both halves.
	This piecewise formula can be alternatively expressed by
		\[\floor*{ \frac{|x+1|}{2} } \]
\end{enumerate}


\end{document}