\documentclass[11pt]{article}

\usepackage{enumerate}
\usepackage{amsmath}
\usepackage{amssymb}
\usepackage{parskip}
\usepackage{mathtools}

\DeclarePairedDelimiter\ceil{\lceil}{\rceil}
\DeclarePairedDelimiter\floor{\lfloor}{\rfloor}

\let\iff\leftrightarrow
\let\imp\rightarrow


\begin{document}

\title{Com S 331 Homework 11}
\author{Adam Hammes $\bullet$ hammesa@iastate.edu}
\date{December 4, 2014}
\maketitle


\section*{Problem 81}

Lemma 1: $f(n)$ grows faster than $f( \log n)$; that is, for any $c \in \mathbb{N}$ there exists $d$ such that $d - \log d > c$. 
It suffices to prove that $\lim _{n \imp \infty} \frac{n}{\log n}= \infty$.
	\begin{align*}
		\lim _{n \imp \infty} \frac{n}{\log n}  &= \lim _{n \imp \infty} n &\text{l'H\^{o}pital's Rule} \\
		&= \infty \ \checkmark
	\end{align*}
Note two things: $T$ is a computable function and $\{ 0^{T(n)} | n \in \mathbb{N} \}$ is infinite.
Consider the TM $M$ defined as follows:
	\begin{itemize}
		\item store $n \in \mathbb{N}$
		\item output $0^{T(n)}$
	\end{itemize}
$M$ demonstrates that
	\begin{align*}
		C( 0^{T(n)} ) &< \log n + c_M + 1 \\
		\intertext{thus} \\
		T( C( 0^{T(n)}) ) &< T( \log n + c_M +1 ).
	\end{align*}
$n$ grows faster than $\log n$, so there exists $a \in \mathbb{N}$ such that $a - \log a > c_M + 1$.
Consider the set $A = \{ 0^b \mid b \in \mathbb{N},\ b \geq a \}$, noting that it is infinite.
Because of how we picked $a$, $\forall x \in A$,
	\begin{align*}
		T(|x|) &< \log |x| \\
		&< |x| \ \checkmark
	\end{align*}


\section*{Problem 82}

Let $c \in \mathbb{N}$. Choose $n \in \mathbb{Z}^+$ such that $C(1^n) > c$ and let $x = 0^n$.
Then
	\begin{align*}
		C(y_{x,1} ^A) &= C(1^n) \\
		&> c + \log 1 \ \checkmark
	\end{align*}


\section*{Problem 83}

Lemma 2: The function $f(n) = (n+1)^2 - n^2$ is strictly increasing.
	\begin{align*}
		(n+1)^2 - n^2 &= n^2 + 2n + 1 + n^2 + 1 \\
		&= 2n +1 \ \checkmark
	\end{align*}

Let $c \in \mathbb{N}$.
Pick $n \in \mathbb{Z^*}$ such that $C( 0^{(n+1)^2 - n^2)}) > c + \log 2$ (we can do this because of Lemma 1) and let $x= 0^{n^2}$.
Then
	\begin{align*}
		C(y _{x, 2} ^A) &= C( 0^{(n+1)^2 - n^2)}) \\
		&> c + \log 2 \ \checkmark
	\end{align*}
Note that we pick the second possible $y$ because $\lambda$ is the first possibility.


\section*{Problem 84}

Let $c \in \mathbb{N}$.
Pick $n \in PRIMES$ such that $C(1^n) > c + 1$ and let $x = 0 ^{(n-1)!}$.
$n$ is prime so gcd( $n$, $(n-1)!$ ) = 1.
Also, $\{2, 3, \dots, n-1\}$ are all multiples of $(n-1)!$ so their gcd with $(n-1)! \neq 1$.
Thus
	\begin{align*}
		y _{x,1} ^{A} &= 1 \\
		y _{x,1} ^{A} &= n \\
	\intertext{and} \\
		C( y _{x,2} ^{A} ) &= C(1^n) \\
		&> c + \log n + \log 2 \ \checkmark
	\end{align*}


\section*{Problem 85}

Let $A$ be any infinite subset of $\{ 0^n 1^n | n \in \mathbb{N} \}$.
It suffices to prove that $A$ is not regular.

Let $c \in \mathbb{N}$. Choose $n \in \mathbb{Z}^+$ such that $C(1^n) > c$ and $0^n 1^n \in A$.
Let $x = 0^n$.
Then
	\begin{align*}
		C(y_{x,1} ^A) &= C(1^n) \\
		&> c + \log 1 \ \checkmark
	\end{align*}


\section*{Problem 86} $B$ is regular by the following regular expression: 

$(00 + 11)^* 01 (0+1)^* $


\section*{Problem 87}

The possible output of the string pairing function $0^{|x|}1xy$ is the set
	\[A = \{0^{|x|}1xy \mid x,y \in \{0,1^*\}\}\]
Let $c \in \mathbb{N}$.
Pick $n$ such that $C(0^n) > c$, and let $x$ = $0^n1$.
Because $0^n$ comes first in the standard enumeration of $\{0,1\}^n$, $y_{x,1} ^A$ = $0^n$ (recall that you can pair $x$ with $\lambda$.
Thus
	\begin{align*}
		y_{x,1} ^A &= C(0^n) \\
		&> c + \log 1
	\end{align*}


\section*{Problem 88}
\begin{enumerate}[(a)]
	\item The last half of a palindrome is uniquely determined by the first half, so the number of palindromes is the number of possible strings that can form the first half. 
	For a string $x$ with even length, the first, the length of the first half is simply 
		\[\frac{|x|}{2} \] 
	and for odd $|x|$, the first half length is
		\[\frac{|x+1|}{2} \]
	since the middle bit is included in both halves.
	This piecewise formula can be alternatively expressed by
		\[\floor*{ \frac{|x+1|}{2} } \]
	which, as to recollect, is the max length of the first half of a string.
	Since a palindrome is deterined by its first half, and there are no restrictions on the form of the palindrom, the number of palindromes of length $n$ is 
		\[2^{\floor*{ \frac{|x+1|}{2} }} \]
	thus
		\[ |PAL \cap \{0,1\}^n| = 2^{\floor*{ \frac{|n+1|}{2} }}\ \checkmark\]

	\item  As proved in lecture, there are infinitely many strings such that their Kolmogorov complexity is greater than their size.
	Let $x$ be any of these strings and $y = x \cdot x^R$.

	Suppose for contradiction that 
		\[C(y) < \floor*{ \frac{|y|+1}{2}} \]
	and consider the following TM $M$:
		\begin{itemize}
			\item store $y$
			\item output the first $\frac{|y|}{2}$ bits of $y$
		\end{itemize}
	which outputs $x$; therefore
		\begin{align*}
			C(x) &< \floor*{ \frac{|y|+1}{2}} + c_M \\
			&< \frac{|y|}{2} + c_M &|y| \text{ is even}
		\end{align*}


\end{enumerate}


\end{document}