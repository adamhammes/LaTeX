\documentclass[11pt]{article}

\usepackage{enumerate}
\usepackage{amsmath}
\usepackage{amssymb}
\usepackage{parskip}
\usepackage{mathtools}

\DeclarePairedDelimiter\ceil{\lceil}{\rceil}
\DeclarePairedDelimiter\floor{\lfloor}{\rfloor}

\let\iff\leftrightarrow
\let\imp\rightarrow


\begin{document}

\title{Com S 331 Homework 11}
\author{Adam Hammes $\bullet$ hammesa@iastate.edu}
\date{December 4, 2014}
\maketitle


\section*{Problem 81}

Let $c \in \mathbb{N}$. Choose $n \in \mathbb{Z}^+$ such that $C(1^n) > c$ and let $x = 0^n$.
Then
	\begin{align*}
		C(y_{x,1} ^A) &= C(1^n) \\
		&> c + \log 1 \ \checkmark
	\end{align*}


\section*{Problem 82}

Lemma 1: The function $f(n) = (n+1)^2 - n^2$ is strictly increasing.
	\begin{align*}
		(n+1)^2 - n^2 &= n^2 + 2n + 1 + n^2 + 1 \\
		&= 2n +1 \ \checkmark
	\end{align*}

Let $c \in \mathbb{N}$.
Pick $n \in \mathbb{Z^*}$ such that $C( 0^{(n+1)^2 - n^2)}) > c + \log 2$ (we can do this because of Lemma 1) and let $x= 0^{n^2}$.
Then
	\begin{align*}
		C(y _{x, 2} ^A) &= C( 0^{(n+1)^2 - n^2)}) \\
		&> c + \log 2 \ \checkmark
	\end{align*}
Note that we pick the second possible $y$ because $\lambda$ is the first possibility.


\section*{Problem 83}

Let $c \in \mathbb{N}$.
Pick $n \in \mathbb{Z^*}$ such that $C(1^n) > c$ and let $x = 0 ^{n-1}$.
Because $n,\ n-1$ are relatively prime and sequential $y _{x,1} ^{A} = 1^n$ so
	\begin{align*}
		C( y _{x,1} ^{A} ) &= C(1^n) \\
		&> c + \log n \ \checkmark
	\end{align*}


\section*{Problem 86} $B$ is regular by the following regular expression: 

$(00 + 11)^* 01 (0+1)^* $


\section*{Problem 88}
\begin{enumerate}[(a)]
	\item The last half of a palindrome is uniquely determined by the first half, so the number of palindromes is the number of possible strings that can form the first half. 
	For a string $x$ with even length, the first, the length of the first half is simply 
		\[\frac{|x|}{2} \] 
	and for odd $|x|$, the first half length is
		\[\frac{|x+1|}{2} \]
	since the middle bit is included in both halves.
	This piecewise formula can be alternatively expressed by
		\[\floor*{ \frac{|x+1|}{2} } \]
	which, as to recollect, is the max length of the first half of a string.
	Since a palindrome is deterined by its first half, and there are no restrictions on the form of the palindrom, the number of palindromes of length $n$ is 
		\[2^{\floor*{ \frac{|x+1|}{2} }} \]
	thus
		\[ |PAL \cap \{0,1\}^n| = 2^{\floor*{ \frac{|n+1|}{2} }}\ \checkmark\]

	\item  As proved in lecture, there are infinitely many strings such that their Kolmogorov complexity.
	Let $x$ be any of these strings, and $y = x \cdot x^R$.

	Suppose for contradiction that 
\end{enumerate}


\end{document}