\documentclass[11pt]{article}
\usepackage{enumerate}
\usepackage{amsfonts}
\usepackage{amsmath}
\usepackage{mathabx}
\usepackage{graphicx}
\usepackage{adjustbox}
\usepackage{parskip}
\usepackage{commath}

\begin{document}

\title{Com S 331 Assignment 4}
\author{Adam Hammes $\bullet$ hammesa@iastate.edu }
\maketitle

\section*{Problem 25}

The starting states can be any possible subset of $Q$, for a total of $\mathbb{P}(Q) = 2^n$ possible different starting states. Similarly, the accepting states are also any possible subset of $Q$, so there are $2^n$ possible different accepting states.

$\Delta$ maps a state/symbol tuple to a subset of $Q$. As discussed in lecture, the number of functions mapping from $a$ items to $b$ items is $n^m$; therefore we simply need to find $b$, the number of subsets of $Q$ and $a$, the number of possible state/symbol pairings.

There are $n$ states and 2 different symbols, thus the number of pairings $a$ is $2n$. As discussed in the first two paragraphs, the number of possible subsets $b= |\mathbb{P}(Q)| = 2^n$. Putting the two numbers together, the number of possible $\Delta$ is $(2^n)^{2n} = 2^{2n^2}$.

Now, these numbers are independent of each other- having a certain $S$ doesn't preclude options for $F$ for example- so to find the total number of NFAs with $Q = \{1, 2, \dots, n \}$ we simply multiply our three numbers together.
	\begin{align*}
		2^n \times 2^n \times 2^{2n^2} &= 2^{2n} \times 2^{2n^2} \\
		&= 2^{2n + 2n^2} \\
		&= 2^{2n(n+1)} \ \checkmark
	\end{align*}
	
\section*{Problem 26}

Prove that there is a language $A \subseteq \{0,1\}^*$ with both of the following properties:
	\begin{enumerate}[(i)]
		\item For all $x \in A$, $|x| \leq 7$.
		\item Every NFA that decides $A$ has more than 8 states.
	\end{enumerate}
	
Using only two symbols, the number of possible strings of length $n$ is $2^n$; allowing $n$ to be anywhere from 0 to 7, we find that there are 
	\[ \sum _{n=0} ^7 2^n = 255 \] 
strings of length seven or less. Our language contains some subset of these strings, so the number of possible languages $A$ is $2^{255}$.

Now, there are two facts we must consider about NFAs. 
	\begin{enumerate}[(1)]
		\item Any $n$ state NFA can be perfectly decided by an NFA with $n+1$ states, by copying the states of the former NFA and adding "junk" states that don't interact with the rest of the NFA. 
		
		Therefore \{languages decided by $n \leq 7$ state NFAs \} $\subset$ \{languages decided by 8 state NFAs\}.
		\item An NFA $N$ decides exactly one language, L($N$). This follows from the definition of what it means for an NFA to solve a language.
	\end{enumerate}

Consider an NFA with $n$ states. By (1), we need only consider for this problem the possible NFAs with exactly $n$ states, since all smaller $n$ is a subset. A formula for the number of such an NFA was found in the previous exercise. If the number of possible NFAs is less than $2^{255}$, than by the Pigeonhole Principle/(2)/(1), the NFAs cannot decide every language $A$. Because the number of NFAs with 8 states is $2^{2*8(8+1)} = 2^{144} < 2^{255}$, the NFAs cannot decide every possible language $A$; therefore we need more than 8 states. $\checkmark$

\section*{Problem 27}
Consider any finite language $A$. I construct an NFA $N = (Q, \Sigma, \Delta, S, F)$ that decides $A$.

First, let's define $Q$. The states of our automata are all the prefixes of the strings in $A$, including $\lambda$ and the strings themselves.

The starting/accepting states reveal a little more about the NFA. Our set of start states is simply the singleton set $\{\lambda\}$; we accept every string (remember that we are representing our states as strings) that is in $A$. This is more simply expressed as $F = A$.

Finally, we arrive at $\Delta$. Our definition of $\Delta$ is actually pretty simple- I'll write it formally then explain it.
	\[ \Delta( q \in Q, a \in \Sigma ) = \{ qa \mid \exists s \in A \text{ such that } qa \sqsubseteq s \} \]
In English, delta of a state (represented by a string)  and a symbol is the concatenation of the two, if that concatenation is a prefix of a string in $A$. If the concatenation is not such a prefix, then the delta of that state and symbol is not defined/$\emptyset$.

With our NFA now defined, let's examine what actually happens. As we feed in a string $s \in A$, our NFA follows down a chain of prefixes in increasing size, starting with $\lambda$, that eventually ends in the state $s$. $s \in A = F$, so the NFA accepts $s$. If we feed in a string that is not in $A$, we'll eventually reach a point where our NFA cannot find a suitable prefix in $Q$, and will jam.

Since our NFA solves $A$, and $A$ is any finite language, any finite language is regular.

\section*{Problem 28}

Let $A$ be $(01)^*\ ||\ (10)^*$. I define $A$ recursively:
	\begin{itemize}
		\item $\lambda \in A$
		\item If $s \in A$, $s \cdot 01 \in A$ and $s \cdot 10 \in A$, where $\cdot$ is the concatenation operator.
	\end{itemize}
	
\section*{Problem 29}
Since $A, B$ are regular sets, there exist DFAs $M_1, M_2$ that solve $A, B$ respectively. The NFA that solves $A\ ||\ B$ will do so by keeping track of two "pebbles" that represent the state $M_1, M_2$, and non-deterministically advancing one pebble as we read in a string.

Our start state is the singleton set $s_1 \times s_2$; similarly, our accepting states are $F_1 \times F_2$. Intuitively, we set the two pebbles down at the start states of $M_1$ and $M_2$, and accept only if both pebbles are in their respective accepting states.

As we read in a string, we need to somehow advance exactly one pebble (since the symbol has to belong to either $A$ or $B$); non-determinism allows us to move exactly the right pebble. Thus $\Delta$ applies the transition function to exactly half of our state (which is an ordered pair), leaving the other part the same. Formally,
	\[ \Delta( (q_1, q_2) \in Q, a \in \Sigma ) = \{ (\delta_1(q_1, a), q_2),\ (q_1, \delta_2(q,a)) \} \]
	
Since each string involved in the shuffle will eventually have all of its symbols read in, each DFA will have its pebble moved all the way to an accepting state and accept the string. If the NFA tries to read an invalid string, one of the DFAs will not reach an accepting state and thus the NFA will reject the string. Thus, the shuffle of any two regular languages is also regular. $\checkmark$

\section*{Problem 30}
\begin{enumerate}[(a)]
	\item $abbbb$
	\item See Figure 30 (last sheet of paper)
\end{enumerate}

\section*{Problem 31}
See Figure 31 (last sheet of paper)

\end{document}