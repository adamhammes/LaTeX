\documentclass[11pt]{article}
\usepackage{enumerate}
\usepackage{amsfonts}
\usepackage{amsmath}
\usepackage{mathabx}
\usepackage{graphicx}
\usepackage{adjustbox}
\usepackage{parskip}
\usepackage{commath}

\begin{document}

\title{Com S 331 Assignment 4}
\author{Adam Hammes $\bullet$ hammesa@iastate.edu }
\maketitle

\section*{Problem 25}

The starting states can be any possible subset of $Q$, for a total of $\mathbb{P}(Q) = 2^n$ possible different starting states. Similarly, the accepting states are also any possible subset of $Q$, so there are $2^n$ possible different accepting states.

$\Delta$ maps a state/symbol tuple to a subset of $Q$. As discussed in lecture, the number of functions mapping from $a$ items to $b$ items is $n^m$; therefore we simply need to find $b$, the number of subsets of $Q$ and $a$, the number of possible state/symbol pairings.

There are $n$ states and 2 different symbols, thus the number of pairings $a$ is $2n$. As discussed in the first two paragraphs, the number of possible subsets $b= |\mathbb{P}(Q)| = 2^n$. Putting the two numbers together, the number of possible $\Delta$ is $(2^n)^{2n} = 2^{2n^2}$.

Now, these numbers are independent of each other- having a certain $S$ doesn't preclude options for $F$ for example- so to find the total number of NFAs with $Q = \{1, 2, \dots, n \}$ we simply multiply our three numbers together.
	\begin{align*}
		2^n \times 2^n \times 2^{2n^2} &= 2^{2n} \times 2^{2n^2} \\
		&= 2^{2n + 2n^2} \\
		&= 2^{2n(n+1)} \ \checkmark
	\end{align*}
	
\section*{Problem 26}

Prove that there is a language $A \subseteq \{0,1\}^*$ with both of the following properties:
	\begin{enumerate}[(i)]
		\item For all $x \in A$, $|x| \leq 7$.
		\item Every NFA that decides $A$ has more than 8 states.
	\end{enumerate}
	
Using only two symbols, the number of possible strings of length $n$ is $2^n$; allowing $n$ to be anywhere from 0 to 7, we find that there are 
	\[ \sum _{n=0} ^7 2^n = 255 \] 
strings of length seven or less. Our language contains some subset of these strings, so the number of possible languages $A$ is $2^{255}$.

Now, there are two facts we must consider about NFAs. 
	\begin{enumerate}[(1)]
		\item Any $n$ state NFA can be perfectly decided by an NFA with $n+1$ states, by copying the states of the former NFA and adding "junk" states that don't interact with the rest of the NFA. 
		
		Therefore \{languages decided by $n \leq 7$ state NFAs \} $\subset$ \{languages decided by 8 state NFAs\}.
		\item An NFA $N$ decides exactly one language, L($N$). This follows from the definition of what it means for an NFA to solve a language.
	\end{enumerate}

Consider an NFA with $n$ states. By (1), we need only consider for this problem the possible NFAs with exactly $n$ states, since all smaller $n$ is a subset. A formula for the number of such an NFA was found in the previous exercise. If the number of possible NFAs is less than $2^{255}$, than by the Pigeonhole Principle/(2)/(1), the NFAs cannot decide every language $A$. Because the number of NFAs with 8 states is $2^{2*8(8+1)} = 2^144 < 2^255$, the NFAs cannot decide every possible language $A$; therefore we need more than 8 states. $\checkmark$

\end{document}