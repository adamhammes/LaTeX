\documentclass[11pt]{article}

\usepackage{enumerate}
\usepackage{amsmath}
\usepackage{amssymb}
\usepackage{parskip}

\let\iff\leftrightarrow
\let\imp\rightarrow

\begin{document}

\title{Com S 331 Homework 12}
\author{Adam Hammes $\bullet$ hammesa@iastate.edu}
\date{December 9, 2014}
\maketitle

Note: $REG$ stands for the set of all regular languages, $DEC$ all decidable languages, $CE$ computable, and $co-CE$ co-computable.


\section*{Problem 89}

Let $A'$ denote the intersection of $0^*10^*$ and $A'$.
Then
	\[A' = \{0^m10^n \mid  m \geq n \}\]
Additionally, let $B$ be defined as follows:
	\[B = \{ \text{rev } a \mid A \} \]
where rev $a \in \{0,1\}^*$ is the reverse of the string.
This language can be alternatively expressed as
	\[ B = \{0^n10^m \mid m \geq n\}\]

Lemma 1: $B$ is not regular.

Let $c \in \mathbb{N}$.
Choose $n \in \mathbb{Z}^+$ such that $C(0^n) > c$, and let $x = 0^n1$.
Then
	\begin{align*}
		C( y^{B} _{x,1} ) &= C( 0^n) \\
		&> c + \log 1\ \checkmark
	\end{align*}

Since $REG$ is closed under intersection (proven in text) and rev (proved in earlier homework), we can say the following:
	\begin{align*}
		A \in REG &\iff A' \in REG \\
		&\iff B \in REG
	\end{align*}
Since $B$ is not regular, $A$ is not regular. $\checkmark$
 



\section*{Problem 90}

Consider $s \in A$.
By definition of $A$, $s$ is of the form $0^kx$ where $\#(0,x) \geq k$ and $k>0$.
Note that $x$ must contain at least one zero.

Let $y$ be $0^{k-1}x$. $x$ has at least one 0, so $0y \in A$.
Also, $0y = s$.
Because $k >0$, such a $y$ exists for every $s \in A$.
Thus we can can express the language $A$ as follows:
	\[ A = \{0x \mid \#(0,x) > 0 \}\]
which is decided by the regular expression
	\[ 0\ (0+1)^*\ 0\ (0+1)^*\]


\section*{Problem 96}

From 95 we know that every c.e. subset of $R$ is finite.
Because $R$ is infinite and a subset of itself, we know that $R \notin CE$.
Since $DEC \subseteq CE$, we can conclude that $R \notin DEC$. $\checkmark$.


\end{document}