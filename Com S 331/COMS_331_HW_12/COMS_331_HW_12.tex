\documentclass[11pt]{article}

\usepackage{enumerate}
\usepackage{amsmath}
\usepackage{amssymb}
\usepackage{parskip}

\let\iff\leftrightarrow
\let\imp\rightarrow

\begin{document}

\title{Com S 331 Homework 12}
\author{Adam Hammes $\bullet$ hammesa@iastate.edu}
\date{December 9, 2014}
\maketitle

Note: $REG$ stands for the set of all regular languages, $DEC$ all decidable languages, $CE$ computable, and $co-CE$ co-computable.


\section*{Problem 89}

Let \textbf{rev} $L \subseteq \Sigma^*$ be as defined on page 302 of the text; that is,
	\[\text{\textbf{rev }} L = \{\text{\textbf{rev }} \mid x \in L\}\]
where \textbf{rev} $x \in \Sigma^*$ is the reverse of the string. Let $B$ = \textbf{rev} $A$.
I prove $B$ is not regular.

Let $c \in \mathbb{N}$.
Choose $n \in \mathbb{Z}^+$ such that $C(0^n) > c$, and let $x = 0^n1$.
The biggest possible value for $k$ in the string is $n$; thus we need at least another $n$ zeroes to get a valid string.
Thus $y^{B} _{x,1} = 0^n10^n$ and 
	\begin{align*}
		C( y^{B} _{x,1} ) &= C( 0^n) \\
		&> c + \log 1\ \checkmark
	\end{align*}

We proved on problem 18 of the homework that if a set $L$ is regular then \textbf{rev} $L$ is also regular.
Since $B$ is not regular, $A$ is not regular by contrapositive. $\checkmark$


\section*{Problem 90}

Consider $s \in A$.
By definition of $A$, $s$ is of the form $0^kx$ where $\#(0,x) \geq k$ and $k>0$.
Note that $x$ must contain at least one zero.

Let $y$ be $0^{k-1}x$. $x$ has at least one 0, so $0y \in A$.
Also, $0y = s$.
Because $k >0$, such a $y$ exists for every $s \in A$.
Thus we can can express the language $A$ as follows:
	\[ A = \{0x \mid \#(0,x) > 0 \}\]
which is decided by the regular expression
	\[ 0\ (0+1)^*\ 0\ (0+1)^*\]


\section*{Problem 91}

Because $A$ is regular, there exists a DFA $M = (Q, \Sigma, \delta, s, F)$ that decides $A$.
Imagine running every string in $A$ through $M$; because $A$ is infinite, an infinite number of strings must ``loop'' inside of $M$, passing through the same state more than once. Call this state $q$; because a string can loop through $q$ once and still be accepted, a string can loop through $q$ any number of times and still be accepted.

Our goal is to have sets $B$ and $C$ represent the strings that go through $q$ an even and odd number of times respectively. The intuition to my approach of differentiating the two is that I have two copies of $Q$, $Q_B$ and $Q_C$. Whenever we go into $q$, we switch sets of states; for instance, if I was in a state in $Q_B$ and was about to enter $q_B$, I would instead go to $q_C$. Thus the two sets of states act as a counter for how many times I've visited $q$.

Formally, I can define two DFAs as follows:
\begin{align*}
	M_B &= (Q \times \{b, c\}, \Sigma, \delta', (s, b), F \times b) \\
	M_C &= (Q \times \{b, c\}, \Sigma, \delta', (s, b), F \times c)
\end{align*}

Let $a \in \{b, c\}$ and $a'$ be whichever character $a$ is not.
We then define the transiton function as follows:
\begin{displaymath}
   \delta'((s \in Q,a), b \in \Sigma) = \left\{
     \begin{array}{lr}
       \delta(s, b) \times a & : \delta(s, b) \neq q\\
       (q, a') & : \delta(s, b) = q
     \end{array}
   \right.
\end{displaymath}

Because a string either passes through $q$ an even or odd number of times (note that not passing through $q$ is the same as passing through an even number of times) and never both, $L(M_B) \cup L(M_C) = A$ and $L(M_B) \cap L(M_C) = \emptyset$. $\checkmark$

\section*{Problem 92}

Let $P_\infty$ be the following property:
	\[ P_\infty = \{ n \mid L(M_n) \text{ is infinite}. \}\] 

Let $M_A$ and $M_B$ be Turing machines, with $M_A$ accepting on all inputs and $M_B$ rejecting on all inputs.
Then L($M_A$) is finite and L($M_B$) is infinite. 
Thus $P_\infty$ is non-trivial.
Since $P_\infty$ is non-trivial, by Rice's theorem $INF$ is undecidable. $\checkmark$



\section*{Problem 93}

Suppose for contradiction that $T$ is computable.
Let $M_K$ be as follows:
\begin{itemize}
	\item input $n \in \mathbb{N}$
	\item let $a = T(n)$
	\item run $M_n(n)$; if in that time $M_n(n)$ halts, ACCEPT else REJECT
\end{itemize}

Because $T(n)$ places an upper limit on how long a halting Turing machine will take to halt, we can definitely tell if $M_n(n)$ halts.
Thus $M_K$ decides the diagonal halting problem, a contradiction. $\checkmark$


\section*{Problem 94}

Consider $R^C$,
	\[ \{x \in \{0,1\}^* \mid C(x) < |x| \}. \]
I prove that $R^C \in CE$, equivalent to $R \in co-CE$.

Let $U$ be a universal Turing machine.
Define $M_R$ as follows:
\begin{itemize}
	\item input $x \in \{0,1\}^* $
	\item let $A$ be $\{0,1\}^{\leq |x|}$
	\item for $i \in \mathbb{Z}^+$ from 1 to $\infty$:
	\begin{itemize}
		\item for $a \in A$, run $U(a)$ for $i$ steps
		\item if $U$ outputs $x$, ACCEPT
	\end{itemize}
\end{itemize}

Consider $x \in R^C$.
Because $C(x) < |x|$, there exists a Turing machine $\pi$ such that $|\pi| < x$ and $U(\pi) = x$.
$M_R$ iterates through every possible encoding of a Turing machine with length less than $|x|$, accepting if and only if the universal Turing machine outputs $x$ on that encoding.
Therefore $L(M_R) = R^C$. $\checkmark$


\section*{Problem 95}

Lemma 1: There are an infinite number of compressible strings.

Proof: Consider $A 0^n \mid n \in \mathbb{N}\}$; for $a \in A$ C($a$) < $\log |n| +c$.
Since $|n|$ grows faster than $\log |n|$, at some point $C(n) < |n|$. $\checkmark$

Assume for contradiction that there is an infinite $CE$ subset of $R$, which I will call $B$.
Let $M_B$ accept $B$.
Consider the following machine $M_f$:
\begin{itemize}
	\item input $n \in \mathbb{N}$
	\item for $i$ from 1 to $\infty$:
	\begin{itemize}
		\item let $c$ = 0
		\item for $j$ from 0 to $i$:
		\begin{itemize}
			\item run $M_B$ on the $j^{\text{th}}$ string in $\{0,1\}^*$ for $i$ steps; if $M_B$ accepts, increment $c$
			\item if $c = n$, output the string we ran $M_B$ on
		\end{itemize}
	\end{itemize}
\end{itemize}

For all $b \in B$, there exists a number of steps such that $M_B$ accepts $b$ in that time; thus $B \subseteq L(M_f)$.
$M_f$ can only output strings that $B$ accepts, so $L(M_f) \subseteq B$ and $B = L(M_f)$.
Thus $M_f$ computes a function $f: \mathbb{N} \imp B$.

By Lemma 1, the number of strings grows faster than the number of incompressible strings.
Because of that, on average the number of strings of a certain length is greater than the number of strings in $B$ with the same length. Thus for any constant $c$ there exists $b \in B$ such that $C(f^{-1}(b)) < C(b)$, a contradiction. $\checkmark$





\section*{Problem 96}

From 95 we know that every c.e. subset of $R$ is finite.
Because $R$ is infinite and a subset of itself, we know that $R \notin CE$.
Since $DEC \subseteq CE$, we can conclude that $R \notin DEC$. $\checkmark$.


\end{document}