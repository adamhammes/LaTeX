\documentclass[11pt]{article}

\usepackage{enumerate}
\usepackage{amsmath}
\usepackage{amssymb}
\usepackage{parskip}

\let\iff\leftrightarrow
\let\imp\rightarrow

\begin{document}

\title{Com S 331 Homework 8}
\author{Adam Hammes $\bullet$ hammesa@iastate.edu}
\date{November 6, 2014}
\maketitle

\section*{Problem 66}

Consider the function $h: \{0,1\}^* \imp \{0,1\}^*$ where $h(x) = x0$.
A TM can compute $h$ by simply appending a 0 to the input, so $h$ is computable.
\begin{itemize}
	\item If $x \in A$, then $h(x) = x0 \in A \sqcup B$ by definition of $\sqcup$. 
	Therefore $x \in A \imp h(x) \in A \sqcup B$.
	\item Every possible output of $h$ ends with a 0.
	The only strings in $A \sqcup B$ that end with a 0 are the strings originally in $A$.
	Therefore $ x \notin A \imp h(x) \notin A \sqcup B$.
\end{itemize}

Since $h$ is computable and $x \in A \iff h(x) \in A \sqcup B$, $A \leq _m A \sqcup B$. $\checkmark$

Now consider the function $g: \{0,1\}^* \imp \{0,1\}^*$ where $g(x) = x1$.
A TM can compute $g$ by simply appending a 1 to the input, to $g$ is computable.
\begin{itemize}
	\item If $x \in B$, then $g(x) = x1 \in A \sqcup B$ by definition of $\sqcup$. 
	Therefore $x \in B \imp g(x) \in A \sqcup B$.
	\item Every possible output of $g$ ends with a 1.
	The only strings in $A \sqcup B$ that end with a 1 are the strings originally in $B$.
	Therefore $ x \notin B \imp g(x) \notin A \sqcup B$.
\end{itemize}

Since $g$ is computable and $x \in B \iff g(x) \in A \sqcup B$, $B \leq _m A \sqcup B$. $\checkmark$.


\section*{Problem 67}

$K^C$ is not c.e. (proved in class and in the text)
Since by Problem 66 $K \leq _M K \sqcup K^C$, $K \sqcup K^C$ is also not c.e. (Theorem 33.3, page 241 of the text). $\checkmark$

Assume for contradiction $K \sqcup K^C$ is co-c.e.
Then the complement of $K \sqcup K^C$ is c.e.
By definition of $\leq _M$, $K^C \leq _M (K \sqcup K^C)^C$, so $K^C$ is c.e., a contradiction.
Therefore $K \sqcup K^C$ is not co-c.e. $\checkmark$


\end{document}