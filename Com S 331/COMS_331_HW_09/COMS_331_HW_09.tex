\documentclass[11pt]{article}

\usepackage{enumerate}
\usepackage{amsmath}
\usepackage{amssymb}
\usepackage{parskip}

\let\iff\leftrightarrow
\let\imp\rightarrow

\begin{document}

\title{Com S 331 Homework 8}
\author{Adam Hammes $\bullet$ hammesa@iastate.edu}
\date{November 6, 2014}
\maketitle

\section*{Problem 66}

Consider the function $h: \{0,1\}^* \imp \{0,1\}^*$ where $h(x) = x0$.
A TM can compute $h$ by simply appending a 0 to the input, so $h$ is computable.
\begin{itemize}
	\item If $x \in A$, then $h(x) = x0 \in A \sqcup B$ by definition of $\sqcup$. 
	Therefore $x \in A \imp h(x) \in A \sqcup B$.
	\item Every possible output of $h$ ends with a 0.
	The only strings in $A \sqcup B$ that end with a 0 are the strings originally in $A$.
	Therefore $ x \notin A \imp h(x) \notin A \sqcup B$.
\end{itemize}

Since $h$ is computable and $x \in A \iff h(x) \in A \sqcup B$, $A \leq _m A \sqcup B$. $\checkmark$

Now consider the function $g: \{0,1\}^* \imp \{0,1\}^*$ where $g(x) = x1$.
A TM can compute $g$ by simply appending a 1 to the input, to $g$ is computable.
\begin{itemize}
	\item If $x \in B$, then $g(x) = x1 \in A \sqcup B$ by definition of $\sqcup$. 
	Therefore $x \in B \imp g(x) \in A \sqcup B$.
	\item Every possible output of $g$ ends with a 1.
	The only strings in $A \sqcup B$ that end with a 1 are the strings originally in $B$.
	Therefore $ x \notin B \imp g(x) \notin A \sqcup B$.
\end{itemize}

Since $g$ is computable and $x \in B \iff g(x) \in A \sqcup B$, $B \leq _m A \sqcup B$. $\checkmark$.


\section*{Problem 67}

$K^C$ is not c.e. (proved in class and in the text)
Since by Problem 66 $K \leq _M K \sqcup K^C$, $K \sqcup K^C$ is also not c.e. (Theorem 33.3, page 241 of the text). $\checkmark$

Assume for contradiction $K \sqcup K^C$ is co-c.e.
Then the complement of $K \sqcup K^C$ is c.e.
By definition of $\leq _M$, $K^C \leq _M (K \sqcup K^C)^C$, so $K^C$ is c.e., a contradiction.
Therefore $K \sqcup K^C$ is not co-c.e. $\checkmark$


\section*{Problem 68}

Assume that $x_A$ is computable. Then there is a function $f_x$ such that for all $f \in \mathbb{N}$, $|f_x(r)-x| \leq 2^{-r}$

There are three possibilities for $A$:
\begin{enumerate}[(1)]
	\item $A$ is finite, in which case $A$ is regular
	\item There exists a $k \in \mathbb{Z}^+$ such that $k, k+1, k+2, \ldots \in A$.
	There are finite number of positive integers $< k$, so we can construct a TM $M$ that decides $A$ by the following process:
	\begin{enumerate}
		\item input $x \in \mathbb{Z}^+$
		\item if $x$ is less than $k$, accept if it is one of the (finitely many) numbers in $A$ less than $k$ and reject otherwise
		\item accept if $x \geq k$
	\end{enumerate}
	\item There does not exist a $k$ as described in (2). This possibility is discussed in the upcoming paragraph(s).
\end{enumerate}

Consider the task of determining if $1 \in A$ given knowledge of $x_A$.
If $x_A > \frac{1}{2}$, then clearly $1 \in A$, since $\sum _{i=2} ^{\infty} = \frac{1}{2} \not> \frac{1}{2}$.
Similarly, if $x_A < \frac{1}{2}$ then $1 \notin A$.

The trickier case is when $x_A = \frac{1}{2}$.
Obviously $A$ could be the set $\{1\}$, so $1 \in A$.
Recall, however, that $\sum _{i=2} ^{\infty} = \frac{1}{2}$, meaning that $A$ could also be $\{2,3,4,\ldots \}$ in which case $1 \notin A$.
Notice that in either case $A$ actually belongs to scenario (1) or (2), and is decidable; We need not concern ourselves with these cases.

Now that we can determine if $1 \in A$, what can we do?
Well, determine if $2 \in A$ for starters.
First, if $1 \in A$ subtract $\frac{1}{2}$ from $x_A$.
Then repeat the process used for $1$, with the magic number being $2^{-2} = \frac{1}{4}$.
If $x_A - 1/2 > 1/4$, then $2 \in A$; else $2 \notin A$.
Note that we don't consider $x_A = 1/4$, as that would mean we are in scenario (1) or (2). 

This process can repeated indefinitely, allowing us to determine if any positive integer is in $A$ if we have knowledge of $x_A$.
However, $f$ gives a bound on $x_A$ that converges at a predictable rate.
To test whether $x_A$ is above or below a certain number, call $f$ with increasingly large numbers until the number we are comparing to is out of the error range.
Because of the way our algorithm works, we can guarantee that the number we are comparing $x_A$ to is not equal to $x_A$ (if it was, we would be in scenario 1 or 2!); therefore the error range will always converge to one side of the number, allowing us to compare it to $x_A$.

With a general idea of how to check membership in $A$, we can finally make a TM $M$ that, given a function $f$ as mentioned in the problem, decides $A$.
\begin{itemize}
	\item input $i \in \mathbb{Z}^+$
	\item let $n$ and $a$ be 0
	\item for $z$ from 1 to $i$:
		\begin{itemize}
			\item compute $f(n) -a$
			\item if $2^{-z}$ falls within the error bound of $f(n)-a$, increment $n$ and go back to the previous step 
			\item if $z =i$, ACCEPT if $2^{-z} < $ the lower bound of $f(n)-a$, else REJECT
			\item if $z \neq i$ and $2^{-z} <$ the lower bound of $f(n)-a$, $a$ += $2^{-z}$
		\end{itemize}
\end{itemize}

Intuition for the machine:

Starting with 1, our machine computes a bound for $x_A$ such that $2^{-1}$ falls outside the bound.
If $2^{-1}$ is less than the bound, then $A$ includes 1, and pretend that $x_A$ is 1/2 less than it was.
Repeat for every integer less than our input, alternating tightening the bound for $x_A$ and testing if numbers fall below or above that bound.

Our machine only accepts an input if it is in $A$, and rejects otherwise, halting either way; therefore $x_A$ being computable implies that $A$ is decidable. $\checkmark$

Now, assume $A$ is decidable.
Then there is a TM $M_A$ that on input $z$, accepts if $z \in A$ else rejects.
Define the machine $M'$ as follows:
\begin{itemize}
	\item input $n \in \mathbb{N}$
	\item let $i$, $j$, $s$ = 0
	\item while $j \neq n$
		\begin{itemize}
			\item run $M_A$ on $i$
			\item if $M_A$ accepts, increment j and add $2^{-i}$ to $s$
			\item increment $i$
		\end{itemize}
	\item output $s$
\end{itemize}

$M'$ finds the $n$ smallest numbers in $A$, then computes $x_{n \text{ smallest elements}}$.
I prove two conditions present in the ``worst case'' scenario (max error):
\begin{itemize}
	\item The maximum error for $M'$ is if every integer larger than the smallest $n$ elements was also present in $A$.
	If this weren't true, then there is an integer $z$ larger than the smaller element not in $A$.
	If $z \in A$, then the error would be $2^{-z}$ bigger.
	Therefore the maximum error for $M'$ is if every integer larger than the smallest $n$ elements is also present in $A$.
	\item The other condition for max error is that the smallest elements used in $M_A$ are $1,2,3, \ldots, n$.
	By the previous condition, every ``bigger'' element is in $A$.
	We can increase the error by ``shifting'' our smallest elements to the left, so they fill the slots $1, 2, 3, \ldots, n$.
	Therefore the maximum error is found when the the smallest elements in $M_A$ are $1,2,3, \ldots, n$.
\end{itemize}

From these conditions we can find the formula for the max error of $M'$:
\begin{align*}
	\sum \limits _{i=n+1} ^{\infty} 2^{-i}&= 1 - \sum \limits _{i=1} ^{n} 2^{-i} \\
	&= 1 - 1 - 2^{-n} \\
	&= 2^{-n}
\end{align*}
Therefore $M'$ computes a function $f: \mathbb{N} \imp \mathbb{Q}$ whose max error with respect to $x_A$ is $2^{-n}$. $\checkmark$


\section*{Problem 69}

Recall that $K$ is c.e.; therefore there exists a machine $M_K$ takes a natural number $z$ and accepts iff $z \in L(M_Z)$.
I define the machine $M'$ as follows:
\begin{itemize}
	\item input $t \in \mathbb{N}$
	\item let s = 0
	\item run $M_k$ for $n$ steps on the first $n$ natural numbers
	\item every time $M_k$ accepts a number $z$, $s$ += $2^{-z}$
\end{itemize}

$M'$ can never be negative, and since it only checks a finite number of Turing Machines, $M'(t \in \mathbb{N}) < x_K$.
Therefore $\forall t \in \mathbb{N},\ 0 \leq M'(t) < x_K$.

As $M'$ accepts more and more numbers, it starts building a ``set'' of numbers.
By increasing the number of steps we run $M'$, this set gets closer and closer to $K$.
Therefore $\lim \limits _{t \imp \infty} M'(t) = x_K$. $\checkmark$


\section*{Problem 71}

The function $f$ described in 68 not only puts an upper and lower bound on the value of $x_K$, you can also deterministically decrease the error by plugging in larger values.
Meanwhile $g$ from 69 provides much less information. It provides only a lower bound of $x_K$; plugging in a larger value may or may not decrease $|g(x) - x_K|$.
Therefore Problem 69 does not contradict 68 because the functions involved compute very different things.

\end{document}