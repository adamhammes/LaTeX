\documentclass[11pt]{article}

\usepackage{enumerate}
\usepackage{amsmath}
\usepackage{amssymb}
\usepackage{parskip}

\let\iff\leftrightarrow
\let\imp\rightarrow

\begin{document}

\title{Com S 331 Homework 8}
\author{Adam Hammes $\bullet$ hammesa@iastate.edu}
\date{October 30, 2014}
\maketitle

\section*{Problem 57}

Since $A, B$ are computably enumerable, there exist machines $M_A, M_B$ such that the machines decide $A$ and $B$ respectively.
I design a TM $N$ that decides $A \cap B$.

Let $N$ be a two-tape TM, with input $x$ on both the top and bottom tapes. 
The top tape simulates $M_A$ and the bottom $M_B$, with the tapes running independently of each other.
$N$ accepts a string if and only if the top and bottom tapes both accept the string. Thus, for all input $x$,
\begin{align*}
	x \in L(N) &\iff N \text{ accepts } x \\
	&\iff M_A \text{ and } M_B \text{ accept } x \\
	&\iff x \in L(M_A),\ L(M_B) \\
	&\iff x \in A, B \\
	&\iff x \in A \cap B \ \checkmark
\end{align*}


\textbf{Note:} for problems 58-63, I will prove equivalence through a ``circle'' of implications, e.g. $(a) \imp (b) \imp (c) \imp \ldots \imp (f) \imp (a)$.

\section*{$(a) \imp (b)$}

Assume $A$ is c.e.
Then there exists a TM that decides $A$, which I will call $M$.
Define the partial function $g: \{0,1\}^* \imp \{0,1\}^* $ as follows:
\[g(x) =
	\begin{cases}
		x & M_A(x) \text{ halts} \\
		\text{undefined} & M_A(x) \uparrow
	\end{cases}
\]
I compute $g$ with the TM $M_g$, which does the following:
\begin{enumerate}
	\item input $x$
	\item run $M$ on $x$
	\item if $M$ halts output $x$
\end{enumerate}

Thus $M_g$ computes a partial function that is defined only on inputs in $A$, equivalent to saying that the domain of $g$ is $A$. $\checkmark$


\section*{$(b) \imp (c)$}

Assume there is a computable partial function $f: \{0,1\}^* \dashrightarrow \{0,1\}^*$ such that $A$ is the domain of $f$.
Without loss of generality, we can assume that it is in fact our function $g$ we defined in the previous problem.
$M_g$ computes $g$, so $g$ is a computable function.
Also, since $g$ is only defined on members of $A$, and $ \forall a \in A,\ g(a) = a$, the range of $g$ is $A$. $\checkmark$


\section*{$(c) \imp (d)$}

If $A$ is empty, then (d) is true; therefore we consider cases where $A \neq \emptyset$.

Assume that there exists a computable partial function $f : \{0,1\}^* \dashrightarrow \{0,1\}^*$ such that $A = $ the range of $f$.
Let $b_n$ be the $n^{\text{th}}$ binary string over $\{0,1\}^*$ in the standard enumeration ($\lambda$, 0, 1, 00, 01, etc.) and $a$ be a string in $A$.
I define the function $h : \mathbb{N} \imp \{0,1\}^*$ as follows:
\[ h(n) =
	\begin{cases}
		b_n & f(b_n) \text{ is defined} \\
		a & f(b_n) \text{ is not defined}
	\end{cases}
\]

Because $h$ is defined directly from the computable function $f$, $h$ is computable.
Also, for all $x\in A$, there exists a $n$ such that $b_n = x$; accordingly, $h(b_n) = x$. 
Therefore range $h \subseteq A$.
Also, for each possible natural number, $h(n)$ maps to a string in $A$, so $A \subseteq $ range $h$ and range $h = A$. $\checkmark$


\section*{$(d) \imp (e)$}
Assume that there is a computable function $f : \mathbb{N} \imp \{0,1\}^*$ such that $A$ = range $f$.

If $A$ is finite, then $(e)$ is true and so is $(d) \imp (e)$.

If $A$ is infinite, we need to find a one-to-one function $m: \mathbb{N} \imp \{0,1\}^*$ such that range $m = A$.
We have a function $h$ that is close, but is not one-to-one.
Since $A$ is infinite, so is range $h$; we just need to ensure that we keep picking new values for the output of $m$.
I denote the output $h$ produces on natural number $n$ as $h_n$.
Keeping these constraints in mind, we can create an algorithm/defition for $m$. 

$m(i \in \mathbb{N}) = $ the smallest $h_n$ such that for $0 \leq k \in \mathbb{N} < j$, $m( k ) \neq h_n$.
In this definition, $k$ serves as ``iterator'' over the output of $m$ on integers $< i$, ensuring that we don't output the same value twice; because $m$ can't output the same value twice, it is one-to-one.

$m$ eventually outputs every $h_n$, and thus eventually every $a \in A$. 
Thus $A \subseteq $ range $m$.
Additionally, since $m$ only can output what $h$ can, range $m \subseteq A$ so range $m = A$. $\checkmark$


\section*{$(f) \imp (a)$}

Assume there is a decidable language $B \subseteq \{0,1\}^*$ such that
	\[A = \{x \in \{0,1\}^* \mid (\exists w \in \{0,1\}^*) \langle x, w \rangle \in B\}\]
Let $M_B$ be a TM that decides $B$, and let $b_0, b_1, \ldots b_i$ be the standard enumeration of $\{0,1\}^*$. 
I prove that there exists a TM $M_A$ that accepts $A$.
Here is the operation of $M_A$:
\begin{enumerate}
	\item input $x$
	\item i = 0
	\item run $M_B$ on $\langle x, b_i \rangle$ 
	\item if $M_B$ accepts, accepts $x$; if $M_B$ rejects, increment $i$ and go back to step 3
\end{enumerate}

For all $x \in A$, there exists a $w$ such that $\langle x, w \rangle \in B$.
Our algorithm simply enumerates through every possible $w$, waiting for $M_B$ to accept or reject the pairing (which it is guaranteed to do, since $B$ is decidable).
$M_A$ will halt and accept a string only if there exists such a pairing.
Therefore $L(M_A) = A$. $\checkmark$


\section*{Problem 64}
I do not know how to solve this problem.

\end{document}