\documentclass[11pt]{article}

\usepackage{enumerate}
\usepackage{amsmath}
\usepackage{amssymb}
\usepackage{parskip}

\let\iff\leftrightarrow
\let\imp\rightarrow

\begin{document}

\title{Com S 331 Homework 8}
\author{Adam Hammes $\bullet$ hammesa@iastate.edu}
\date{October 30, 2014}
\maketitle

\section*{Problem 57}

Since $A, B$ are computably enumerable, there exist machines $M_A, M_B$ such that the machines decide $A$ and $B$ respectively.
I design a TM $N$ that decides $A \cap B$.

Let $N$ be a two-tape TM, with input $x$ on both the top and bottom tapes. 
The top tape simulates $M_A$ and the bottom $M_B$, with the tapes running independently of each other.
$N$ accepts a string if and only if the top and bottom tapes both accept the string. Thus, for all input $x$,
\begin{align*}
	x \in L(N) &\iff N \text{ accepts } x \\
	&\iff M_A \text{ and } M_B \text{ accept } x \\
	&\iff x \in L(M_A),\ L(M_B) \\
	&\iff x \in A, B \\
	&\iff x \in A \cap B \ \checkmark
\end{align*}


\textbf{Note:} for problems 58-63, I will prove equivalence through a ``circle'' of implications, e.g. $(a) \imp (b) \imp (c) \imp \ldots \imp (f) \imp (a)$.

\section*{$(a) \imp (b)$}

Assume $A$ is c.e.
Then there exists a TM that decides $A$, which I will call $M$.
Define the partial function $g: \{0,1\}^* \imp \{0,1\}^* $ as follows:
\[g(x) =
	\begin{cases}
		0 & M_A(x) \text{ halts} \\
		\text{undefined} & M_A(x) \uparrow
	\end{cases}
\]
I compute $g$ with the TM $M_g$. $M_g$ does the following:
\begin{enumerate}
	\item input $x$
	\item run $M$ on $x$
	\item if $M$ halts output 0 
\end{enumerate}

Thus $M_g$ computes a partial function that is defined only on inputs in $A$, equivalent to saying that the domain of $g$ is $A$. $\checkmark$

\end{document}