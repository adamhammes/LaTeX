\documentclass[11pt]{article}

\usepackage{enumerate}
\usepackage{amsmath}
\usepackage{amssymb}
\usepackage{parskip}

\let\iff\leftrightarrow
\let\imp\rightarrow

\begin{document}

\title{Com S 331 Homework 8 Remake}
\author{Adam Hammes $\bullet$ hammesa@iastate.edu}
\date{November 13, 2014}
\maketitle

\section*{Problem 64}

Assume that $A$ is decidable. 
Then there exists a TM $M_A$ that decides $A$.
Prove that one of the following is true:
\begin{enumerate}[(a)]
	\item $A$ is finite
	\item $A$ is infinite, and there is a computable function $f: \mathbb{N} \imp \{0,1\}^*$ such that range(f) = A and each $f(n)$ comes strictly before $f(n+1)$ in the standard enumeration of $\{0,1\}^*$.
\end{enumerate}

$A$ is either finite or infinite.
If it is finite, then we have satisfied (1) and the implication is true.
If $A$ is infinite, then define the TM $M_f$ as follows:
\begin{itemize}
	\item input $n \in \mathbb{N}$
	\item let $c$ be 0
	\item for $i$ from 0 to $\infty$:
	\begin{itemize}
		\item let $s$ be the $i^{\text{th}}$ string in the standard enumeration of $\{0,1\}^*$
		\item if $M_A$ accepts $s$, increment $c$
		\item if $c > n$ output $s$
	\end{itemize}
\end{itemize}

$M_f$ iterates through the standard enumeration of $\{0,1\}^*$, running $M_A$ on each until we have found $n$ members of $A$.
Because of this iteration, $f( n+ 1) > f(n)$.
Also, since $M_f$ only outputs strings that $M_A$ accepts, and eventually outputs every string in $A$, range($f$) = $A$.
Therefore $A$ is decidable implies (2). $\checkmark$

Now, assume there exists a computable function $f: \mathbb{N} \imp \{0,1\}^*$ such that range(f) = A and each $f(n)$ comes strictly before $f(n+1)$ in the standard enumeration of $\{0,1\}^*$.
Let $M_f$ be the machine that computes $f$.
Define the machine $M_A$ as follows:
\begin{itemize}
	\item input $x \in \{0,1\}^*$
	\item for $i$ from 0 to $\infty$:
	\begin{itemize}
		\item $s := M_f(i)$
		\item if $s == x$ ACCEPT
		\item if $s$ comes after $x$ in the standard enumeration, REJECT
	\end{itemize}
\end{itemize}

Consider a string $x \in \{0,1\}^*$.
$M_f$ outputs $x$ iff $(\exists n \in \mathbb{N})\ f(n) = x$.
$M_A$ eventually tries every natural number, and can only output what $f$ outputs, so $L(M_A)$ = range($f$).
Also, because eventually $f(n) > x$ and $f(n+1) > f(n)$, $M_A$ can tell definitively if $x \notin A$ by rejecting if $M_f(i)$ has ``passed by'' $x$.
Therefore $M_A$ decides $A$. $\checkmark$


\end{document}