\documentclass[11pt]{article}
\usepackage{enumerate}
\usepackage{amsfonts}
\usepackage{amsmath}
\usepackage{mathabx}
\usepackage{graphicx}
\usepackage{adjustbox}


\begin{document}

\title{Com S 331 Assignment 1}
\author{Adam Hammes $\bullet$ hammesa@iastate.edu}
\maketitle

\section*{Problem 1}

$01$ is in $A$; therefore $s = 0101$ is in $A^*$. However, since $s$ alternates between 0 and 1, it is clearly not in $A$. Therefore $A^* \neq A$.

\section*{Problem 2}
$B^*$ is  the union of all $B^n, n \in \mathbb{N}$, which in turn is defined recursively as
	\begin{itemize}
		\item 		$B^0	= \{\lambda\}$
		\item		$B^{n+1}	 = BB^n$
	\end{itemize}
I prove by induction that $B^n, n \in \mathbb{N} \subseteq B$.

\begin{description}
	\item[Base Case:] Base: $n = 0$. $B^0$ has one element, $\lambda$. $\lambda$ has the same amount of zeroes and ones (none), and is therefore in $B$.
	
	\item[Inductive step:] Assume $B^n \subseteq B$. Prove that $B^{n+1}$ $\subseteq B$\\
	Let $x$ be any element in $B^n$, and $y$ an element in $B$. By the inductive hypothesis, $\#(0,x) =\#(1,x)$ and by definition of $B$ $\#(0,y) = \#(1,y)$. The concatenation $xy$ also shares this property, since $\#(0,xy)$ and $\#(1, xy)$ are the sum of the zeroes and ones in $x,y$ and are therefore equal. Therefore $B^{n+1} \subseteq B$ and by induction $B^* \subseteq B$
\end{description}
By concatenating any string in $B$ with $\lambda$ (also found in $B$), you find the same string in $B^*$; therefore $B \subseteq B^*$. Since $B \subseteq B^*$ and $B^* \subseteq B$, $B = B^*$




\end{document}