\documentclass[11pt]{article}
\usepackage{enumerate}
\usepackage{amsfonts}
\usepackage{amsmath}
\usepackage{mathabx}
\usepackage{graphicx}
\usepackage{adjustbox}


\begin{document}

\title{Com S 331 Assignment 1}
\author{Adam Hammes $\bullet$ hammesa@iastate.edu}
\maketitle

\section*{Problem 1}

$01$ is in $A$; therefore $s = 0101$ is in $A^*$. However, since $s$ alternates between 0 and 1, it is clearly not in $A$. Therefore $A^* \neq A$. $\checkmark$

\section*{Problem 2}
$B^*$ is  the union of all $B^n, n \in \mathbb{N}$, which in turn is defined recursively as
	\begin{itemize}
		\item 		$B^0	= \{\lambda\}$
		\item		$B^{n+1}	 = BB^n$
	\end{itemize}
I prove by induction that $B^n, n \in \mathbb{N} \subseteq B$.

\begin{description}
	\item[Base Case:] Base: $n = 0$. $B^0$ has one element, $\lambda$. $\lambda$ has the same amount of zeroes and ones (none), and is therefore in $B$.
	
	\item[Inductive step:] Assume $B^n \subseteq B$. Prove that $B^{n+1}$ $\subseteq B$\\
	Let $x$ be any element in $B^n$, and $y$ an element in $B$. By the inductive hypothesis, $\#(0,x) =\#(1,x)$ and by definition of $B$ $\#(0,y) = \#(1,y)$. The concatenation $xy$ also shares this property, since $\#(0,xy)$ and $\#(1, xy)$ are the sum of the zeroes and ones in $x,y$ and are therefore equal. Therefore $B^{n+1} \subseteq B$ and by induction $B^* \subseteq B$
\end{description}
By concatenating any string in $B$ with $\lambda$ (also found in $B$), you find the same string in $B^*$; therefore $B \subseteq B^*$. Since $B \subseteq B^*$ and $B^* \subseteq B$, $B = B^*$. $\checkmark$

\section*{Problem 3}

We prove through induction.
\begin{description}
	\item[Base Case:] $n= 0$
		\begin{align*}
				\dfrac{1}{1} &\leq 2 - \dfrac{1}{1}\\
				1 &\leq 1\ \checkmark
		\end{align*}
		
	\item[Inductive Step:] Assume $\sum\limits_{k=1}^n \dfrac{1}{k^2} \leq 2 - \dfrac{1}{n}$. Prove $\sum\limits_{k=1}^{n+1} \dfrac{1}{k^2} \leq 2 - \dfrac{1}{n+1}$. \\
	We can subtract $\sum\limits_{k=1}^{n} \dfrac{1}{k^2}$ from each side to reach the following:


\end{description}

\section*{Problem 4}

$\forall x \in A^*$, $\lambda x = x$ and $ \lambda x \in A^{**}$. Therefore $A^* \subseteq A^{**}$.\\\\
Consider any element $a \in A^{**}$. $a$ is the concatenation of strings $b_0, b_1,...$, $b_i \in A^*$. Each string $b_i$ is in turn composed of substrings $c_{i,1}, c_{i,2}, ...$, where $c_{i,j} \in A$. Therefore $a$ is composed solely of strings found in $A$. Since $a$ is any string in $A^{**}$, and $A^*$ contains every possible concatenation of the strings in $A$, $A** \subseteq A*$.\\\\
Since $A^* \subseteq A^{**}$ and $A** \subseteq A*$, $A^* = A^{**}$. $\checkmark$

\section*{Problem 5}

I prove the contrapositive, $ S \subsetneq T \implies S^* \neq T^*$.\\\\
For $S \subsetneq T$ to be true, there is an element $a \in S$ that is not in $T$. This element is in $S^*$, but not in $T^*$. Therefore $S^* \neq T^*$. $\checkmark$


\section*{Problem 6}
$ S = \{0, 1, 11\}$, $T = \{0, 1\}$


\end{document}