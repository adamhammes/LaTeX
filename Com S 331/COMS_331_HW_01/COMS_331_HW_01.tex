
\documentclass[11pt]{article}
\usepackage{enumerate}
\usepackage{amsfonts}
\usepackage{amsmath}
\usepackage{mathabx}
\usepackage{graphicx}
\usepackage{adjustbox}
\usepackage{parskip}

\begin{document}

\title{Com S 331 Assignment 1}
\author{Adam Hammes $\bullet$ hammesa@iastate.edu}
\maketitle

\section*{Problem 1}

$01$ is in $A$; therefore $s = 0101$ is in $A^*$. However, since $s$ alternates between 0 and 1, it is clearly not in $A$. Therefore $A^* \neq A$. $\checkmark$

\section*{Problem 2}
$B^*$ is  the union of all $B^n, n \in \mathbb{N}$, which in turn is defined recursively as
	\begin{itemize}
		\item 		$B^0	= \{\lambda\}$
		\item		$B^{n+1}	 = BB^n$
	\end{itemize}
I prove by induction that $B^n, n \in \mathbb{N}, \subseteq B$.

\begin{description}
	\item[Base Case:] Base: $n = 0$. $B^0$ has one element, $\lambda$. $\lambda$ has the same amount of zeroes and ones (none), and is therefore in $B$.
	
	\item[Inductive step:] Assume $B^n \subseteq B$. Prove that $B^{n+1}$ $\subseteq B$\\
	Let $x$ be any element in $B^n$, and $y$ an element in $B$. By the inductive hypothesis, $\#(0,x) =\#(1,x)$ and by definition of $B$ $\#(0,y) = \#(1,y)$. The concatenation $xy$ also shares this property, since $\#(0,xy)$ and $\#(1, xy)$ are the sum of the zeroes and ones in $x,y$ and are therefore equal. Therefore $B^{n+1} \subseteq B$ and by induction $B^* \subseteq B$
\end{description}
By concatenating any string in $B$ with $\lambda$ (also $\in B$), you find the same string in $B^*$; therefore $B \subseteq B^*$. Since $B \subseteq B^*$ and $B^* \subseteq B$, $B = B^*$. $\checkmark$

\section*{Problem 3}

We prove through induction.
\begin{description}
	\item[Base Case:] $n= 1$
		\begin{align*}
				\dfrac{1}{1} &\leq 2 - \dfrac{1}{1}\\
				1 &\leq 1\ \checkmark
		\end{align*}
		
	\item[Inductive Step:] Assume $\sum\limits_{k=1}^n \dfrac{1}{k^2} \leq 2 - \dfrac{1}{n}$. Prove $\sum\limits_{k=1}^{n+1} \dfrac{1}{k^2} \leq 2 - \dfrac{1}{n+1}$. \\
	We subtract $\sum\limits_{k=1}^{n} \dfrac{1}{k^2}$ from both sides to reach the following:
		\begin{equation}
		\dfrac{1}{(n+1)^2} \leq 2 - \dfrac{1}{n+1} - \sum\limits_{k=1}^{n} \dfrac{1}{k^2}
		\end{equation}
		Also, the following can be reached from the inductive hypothesis by diving by $-1$:
		\begin{equation*}
		-\sum\limits_{k=1}^n \dfrac{1}{k^2} \geq \dfrac{1}{n} -2
		\end{equation*}
		Since the right hand side of $(2)$ is less than $-\sum\limits_{k=1}^n \dfrac{1}{k^2}$, we can substitute it into $(1)$ and still hold the inequality.
		
	\begin{align*}
	\dfrac{1}{(n+1)^2} &\leq 2 - \dfrac{1}{n+1} + \dfrac{1}{n} -2 \\
	\dfrac{1}{(n+1)^2} &\leq - \dfrac{1}{n+1} + \dfrac{1}{n}	\\
	\dfrac{1}{(n+1)^2} &\leq \dfrac{1}{n^2 +1}\\
	1 &\leq \dfrac{(n+1)^2}{n^2 +1}\\
	1 &\leq \dfrac{n^2 + 2n + 1}{n^2 + 1}
	\end{align*}
	
	Because of the $2n$ term, this inequality is true for $\mathbb{Z^+}$, concluding the inductive step. $\checkmark$


\end{description}

\section*{Problem 4}

$\forall x \in A^*$, $\lambda x = x$ and $ \lambda x \in A^{**}$. Therefore $A^* \subseteq A^{**}$.

Consider any element $a \in A^{**}$. $a$ is the concatenation of strings $b_0, b_1,...$, $b_i \in A^*$. Each string $b_i$ is in turn composed of substrings $c_{i,1}, c_{i,2}, ...$, where $c_{i,j} \in A$. Therefore $a$ is composed solely of strings found in $A$. Since $a$ is any string in $A^{**}$, and $A^*$ contains every possible concatenation of the strings in $A$, $A^{**} \subseteq A*$.


Since $A^* \subseteq A^{**}$ and $A^{**} \subseteq A*$, $A^* = A^{**}$. $\checkmark$

\section*{Problem 5}

I prove the contrapositive, $ S \subsetneq T \implies S^* \neq T^*$.


For $S \subsetneq T$ to be true, there is an element $a \in S$ that is not in $T$. This element is in $S^*$, but not in $T^*$. Therefore $S^* \neq T^*$. $\checkmark$

\section*{Problem 6}
$ S = \{0, 1, 11\}$, $T = \{0, 1\}$

Since $T \subset S$, $T^* \subseteq S^*$. Also, since the only difference between $S$ and $T$ is $11$, which can be replicated by repeating a $1$, $S^* \subseteq T^*$ and by implication $S^* = T^*$. $\checkmark$

\section*{Problem 7}
There are an infinite number of strings $w$ such that $ww \sqsubseteq s$. Since two different strings of the same length cannot both be prefixes of the same string, there is no cap on the length of $w$.

Consider a string $w \sqsubseteq s$.  The first bits $b_1 - b_{|w| -1}$ are $w$, and so are the bits $b_{w} - b_{w-1}$; that is to say, the first $w$ bits are the same as the next $w$ bits. The probability of this resulting from independent coin tosses is directly proportional to the size of $w$:
	\begin{align*}
	P = \dfrac{1}{2^{|w|}}
	\end{align*}
Since for any finite length $a\ \exists w \sqsubseteq s$ such that $|w| > a$, this probability can be further reduced to:
	\begin{align*}
	P &= \dfrac{1}{2^{\infty}} \\
	&=0\ \checkmark
	\end{align*}
	

\section*{Problem 8}
Consider any infinite binary string $s$, composed of finite strings $w_0, w_1, w_2  \dots \in W$. Since $|s| = \infty$, $|W| = \infty$. Because of this, there are either an infinite number of red $w$ or an infinite number of blue $w$ (or both).

Define a string $p \in \{0,1\}^{\infty}$ to be prefix-red if $\exists \ a \sqsubseteq p $ such that $\chi(a) =$ red and the infinite string after $a$ is also prefix-red; that is to say, any prefix-red string is composed of an infinite sequence $w_1, w_2, \dots$, $\chi(w_i) =$ red. Also define a prefix-blue string to be the same, but with the color blue.

Now, for every infinite binary sequence, there are an infinite number of red substrings and/or an infinite number of blue substrings. Assume for contradiction that a string with an infinite number of one color ($c$) substrings does not end in a prefix-$c$ tail. This would imply that at some point we are unable to pick another prefix that is $c$ color; however, since the sequence contains an infinite number of $c$ strings that is impossible. Therefore $\forall s \in \{0,1\}^{\infty}$, $s$ is either prefix-red or prefix-blue.

Because of this, for any $s$ we simply pick a prefix $w_0$ such that the resulting tail is prefix-red (if $s$ has an infinite number of red substrings) else prefix-blue. $s$ can now be written as the sequence $w_0, w_1, w_2  \dots $ where $w_1, w_2, \dots$ are the same color and are the prefixes to the infinitely many prefix-red/blue tails. $\checkmark$

\end{document}

\section*{Problem 6}
Let $G$ denote the probability of guessing the password. I also assume that the spyware chooses random passwords to test.

\begin{enumerate}(a)
	\item $G = \dfrac{10^6}{P(26, 6)} = \dfrac{10^7}{26!/(6!} \approx 0.00603"$

\end{enumerate}