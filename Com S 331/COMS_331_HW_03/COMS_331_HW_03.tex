\documentclass[11pt]{article}
\usepackage{enumerate}
\usepackage{amsfonts}
\usepackage{amsmath}
\usepackage{mathabx}
\usepackage{graphicx}
\usepackage{adjustbox}
\usepackage{parskip}
\usepackage{commath}

\begin{document}

\title{Com S 331 Assignment 3}
\author{Adam Hammes $\bullet$ hammesa@iastate.edu }
\maketitle

\section*{Problem 19}
Consider a DFA $M$. I define a "good" string $a$ to be a string for which $ \exists\ b \in \Sigma^*$ such that $M$ accepts $ab$; that is, there exists a path in the DFA from $a$ to an accepting state. Note that by this definition, $\lambda$ is a good string for any DFA with $F \neq \emptyset$.

Now, consider a string $s$ that is rejected by a $M$. While reading $s$, after some specific bit it became impossible for $M$ to accept $s$; that is, everything before the bit made up a good string, which concatenated with $b$ makes up a string that is "bad". I call this bit a mistake.

The set $N_2(A)$ contains all the strings that are at most Hamming distance 2 away from $A$. With respect to the DFA that decides $A$, which I will call $M = (Q, \Sigma = \{0,1\}, \delta, s, F)$, the strings in $N_2(A)$ can have at most two mistakes; therefore the NFA that decides $N_2(A)$ can be very similar to $M$, but with the additional task of counting the number of mistakes. In the following paragraphs I will describe this NFA, which I will call $N$.  

The states in $N$ have two parts; first, a state $q \in Q$ which keeps track of what state we are in; second, a number $n \in \{0, 1, 2\}$ that keeps track of how many mistakes we have made. The alphabet is the same as in $M$, $\Sigma$.

The transition function is very similar to that of $M$. Consider the two possibilities when in state $(q, n)$ and reading a bit $b$:
\begin{enumerate}
	\item $b$ is not a mistake
	\item $b$ is a mistake
\end{enumerate}

If $b$ is not a mistake, then our job is easy; we simply go the state $M$ would if we had read $b$ while in $q$, and the number of mistakes; put formally, when not reading a mistake \[ \Delta( (q, n), b ) = (\delta( q, b), n) \text{ if $b$ is not a mistake} \].

If $b$ is a mistake , our job is a little complex, and depends on what $n$ is. If $n$ is 2, then we have already made two mistakes and $N$ should reject whatever string we've been reading. In an NFA, this is accomplished by simply not defining the transition. If $n$ is one or two, we can tolerate a mistake. The question is, what state do we go to? 

Obviously, $n$ will be increased by one to account for the mistake we just read. Perhaps more subtly, our transition should pretend that we had read the opposite value of the mistake; this allows us to continue on our job of following $M$ while keeping track of how mistakes we've made. Borrowing the notation $\overline{b}$ to mean the logical negation of $b$, we can extend our $\Delta$ definition as follow:
	\[ \Delta( (q, n), b) = (\delta(q, \overline{b}), n+1) \text{ if $b$ is a mistake and $n \neq$ 2}\]
	
Finally, we arrive at the last two parts of our NFA, the starting and accepting states. $N$ will start where $M$ did, and with zero mistakes; thus, $S = \{ (s, 0) \}$. Likewise, $N$ will accept the states $M$ did, as long as haven't made too many mistakes, so the accepting states are $(f, n )\ |\ f \in F $ and $ n \in \{0, 1, 2\}$.

Let us now examine $N$ intuitively. If a string is in $N_2(A)$, $N$ will find the related string in $A$ and follow it, using the mistake transitions at most twice. If a string is not in $N_2(A)$, then $N$ will "jam" on some state after it runs out of mistakes. Because $N$ decides any $N_2(A)$ such that $A$ is regular, if $A$ is regular, $N_2(A)$ is regular. $\checkmark$

\section*{Problem 20}
I prove through induction.
\begin{description}
	\item [Base Case:] $n = 0$
	There are only two possible outputs for $\hat{\delta(u, v)}$; if $u,\ v$ are distinct, then it's not possible for the automaton to reach $u$ and therefore $\hat{\delta(u,v)}= \emptyset$. If $u = v$, then the automaton can reach $v$ simply by reading the empty string. Therefore the transition matrix is $\{\lambda \}$ if $u =v$, else $\emptyset$, which is the definition of $I$, which is $A^0$. $\checkmark$
	
	\item [Inductive Step:] $(A^{n+1})_{uv} =$
	\begin{align*}
		&(A^n)_{uw} * (A_{wv}) &\text{def. of $A^{n+1}$} \\
		&\{x \in \Sigma^* \mid\ |x| = n \wedge \hat{\delta(u,x)} = w\} * \{a \in \Sigma \mid \delta(w, a) = v\} &\text{ind. hypothesis}\\
		&\{ xa \in \Sigma^* \mid |xa| = n+1 \wedge \hat{\delta(u, xa)} = v \} &\text {concat., def of $\hat{\delta}$}
	\end{align*}
	Defining $b$ to be the concatenation of $x, a$, we arrive at the following:
	\[\{b \in \Sigma^* \mid\ \abs{b}= n +1 \wedge \hat{\delta(u, b) = v} \}\ \checkmark \]
	

\end{description}




\section*{Problem 21}
Using our results from the previous problem, we can say that
	\[ \bigcup _{n \geq 0} (A^n)_{uv} \]
is the set of all strings $a$ with finite length such that $\hat{\delta}(u, a) = v $. When we let $v$ be any accepting state and $u$ the start state, as in 
	\[ \bigcup _{t \in F} (A^*)_{st} \text{ ,} \]
we are describing all the possible strings $a$ such that $\hat{ \delta }(s, a) \in F$, which is by definition the set of strings that the automaton accepts. $\checkmark$

\end{document}