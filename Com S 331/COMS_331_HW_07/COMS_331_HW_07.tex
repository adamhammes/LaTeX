\documentclass[11pt]{article}

\usepackage{enumerate}
\usepackage{amsmath}
\usepackage{amssymb}
\usepackage{parskip}

\begin{document}

\title{Com S 331 Homework 7}
\author{Adam Hammes $\bullet$ hammesa@iastate.edu}
\date{October 22, 2014}
\maketitle

\section*{Problem 49}

Let $A$ be any infinitely large subset of $\{a^nb^n \mid n \geq 0\}$, $M = (Q, \Sigma, \delta, s, F)$ be any DFA. It suffices to prove that $L(M) \neq A$

 Since $A$ is infinitely large, $\exists s=a^jb^j \in A,$ where $j > |Q|$. Since $j > |Q|$, by the Pigeonhole Principle $M$ passes through a state $q \in Q$ at least twice while reading $a^j$. Let $a^k$ be the string that travels the same path as $a^j$, but skipping the states from the first $q$ up to and including the second $q$. Note that since we are always removing at least one state, the second $q$, $k < j$.

Because $a^k,\ a^j$ still end in the same state, $\hat{\delta}(a^k) = \hat{\delta}(a^j)$ which implies that by definition of $\hat{\delta}$ that $\hat{\delta}(a^jb^j) = \hat{\delta}(a^kb^j)$, so $M$ accepts both $a^ja^j$ and $a^ka^j$. However, $a^ka^j$ is clearly not in $A$ since $k \neq j$; therefore $L(M) \neq A$. $\checkmark$




\section*{Problem 50}

\begin{itemize}
	\item $Q = \{ s,\ a,\ b,\ c,\ t,\ r \}$
	\item $\Sigma = \{0,\ 1\}$
	\item $\Gamma = \{ \vdash,\ 0,\ 1,\ $\textvisiblespace$\}$
\end{itemize}

$\delta$ is defined according to the following table:

\begin{tabular}{ c | c | c | c | c }
			& $\vdash$ 			& 0 					& 1 				& \textvisiblespace \\
			\hline
	s		& a, $\vdash$, R 	& -					& -				& - \\
	a		& - 						& r, -, -				& b, 1, R		&  r, -, -\\
	b		& -						& b, 0, r			& b, 1, R		& c, \textvisiblespace , L \\
	c		& -						& t, -, -				& r, -, -			& -
\end{tabular}


\section*{Problem 51}

\begin{itemize}
	\item $Q = \{ s,\ a,\ b,\ c,\ t,\ r \}$
	\item $\Sigma = \{0,\ 1\}$
	\item $\Gamma = \{ \vdash,\ 0,\ 1,\ $\textvisiblespace$\}$
\end{itemize}

$\delta$ is defined according to the following table:


\begin{tabular}{ c | c | c | c | c }
			& $\vdash$				& 0 					& 1				& \textvisiblespace \\
		\hline
		s	& a, $\vdash$, R		& -					& -				& - \\
		a	& -							& b, 0, R			& a, 1, R		& r, -, - \\
		b	& -							& b, 0, R			& c, 1, R		& r, -, - \\
		c	& -							& t, -, -				& a, 1, R		& r, -, -
\end{tabular}


\section*{Problem 52}

\begin{itemize}
	\item $Q = \{ s,\ a,\ b,\ t,\ r \}$
	\item $\Sigma = \{0,\ 1\}$
	\item $\Gamma = \{ \vdash,\ 0,\ 1,\ $\textvisiblespace$\}$
\end{itemize}

$\delta$ is defined according to the following table:


\begin{tabular}{ c | c | c | c | c }
			& $\vdash$				& 0 					& 1				& \textvisiblespace \\
		\hline
		s	& a, $\vdash$, R		& -					& -				& - \\
		a	& -							& b, 0, R			& a, 1, R		& r, -, - \\
		b	& -							& r, -, -				& b, 1, R		& t, -, - \\
\end{tabular}


\section*{Problem 53}

Note for the next three problems: When I say leftmost character, I am excluding the left end-marker.

Intuition: First, we check if there are 0 a's, in which case we reject, or 1 a, where we accept. Then our machine $M$ will erase all a's but the first, additionally marking the rightmost one with a b. 

Now we begin the loop; make the last a an \~a , letting us keep track of the original length. Then copy all the a's on the tape to the space right of \~a, up to and including \~a, effectively doubling the number of a's on the tape. If we try to overwrite our b with an \~a, accept because the original input had a power of two number of a's; if we overwrite b with a regular a, reject because the original input was not a power of two. $\checkmark$

To do the copying mentioned in the previous paragraph, use an additional marker to keep track of what character is being copied; this marker starts on the first character, and moves one character right after each copy. The last copy is made when the marker is at the end of the original input. I make note of this because I will use this copying again in the next problem.


\section*{Problem 54}

Intuition: First, accept if the input is empty spaces or is exactly one a. If we don't accept, replace the last a on the tape with a b, and put a "copy marker" and an "add marker" on the first a.

Now we begin our loop, which is actually a double loop. Say that there are initially $n$ a's. The inner loop uses the copy process described in 53 to append an additional $n$ a's to the tape, using the copy marker to do so. Once the $n$ copies have been made we exit to the outer loop.

Outer loop: Move the add marker one character right. If the add marker is on the last character, replace every marked character/b with an a then accept. Note that we have done $n-1$ copies of $n$ a's at this point.


At the point we have accepted/exited the outer loop, we have written an additional $n$ a's $n-1$ time. When we take into account the original $n$ a's, we have $n * (n-1) +n = n^2$ a's on the tape. $\checkmark$.


\section*{Problem 55}

Subroutine: Say that $a^j \# a^k$ is written on the tape. This algorithm will flip the sides if necessary to ensure that more a's are on the left.

Mark the leftmost unmarked a, and the leftmost unmarked a that is to the right of $\#$. Continue to do so until we attempt to mark a $\#$ or a \textvisiblespace. There are three cases:

\begin{itemize}
	\item We tried to mark both characters at the same time. The strings are equal length, so we do nothing.
	\item We tried to mark just the hash tag. The left side side is smaller, so we swap the hashtag and the closest marked a on the right, effectively swapping the lengths of the two strings.
	\item We tried to mark the space; right side is smaller, which is good. Do nothing.
\end{itemize}


Our Turing Machine operates as follows:

If the leftmost character is $\#$, delete it, shift all the a's one space left, and then accept. If we don't accept, use the subroutine described earlier to make sure the left input has more a's. 

Now the tape reads $\vdash \ a^j \# \mid a^k$ where $j >= k$.


Then, using a marker on both the input sides, replace a's on the left side with spaces until the right marker hits a space. Then shift all the non-space characters to the left so that their are no spaces between $\vdash$ and the first a, effectively $k$ a's from the left side.

Repeat this deletion/shifting until we attempt to delete the $\#$; if this happens, we have subtracted too many times, and we need to undo the deletion by filling in all spaces between $\vdash$ and $\#$ with a's. At this point our tape reads $\vdash \  a^{j\ mod\ k}\ \#\ a^{k}$, and we can repeat the loop.

To sum up the algorithm, our machine recursively takes the modulus of the two inputs, until doing so results in $j=0$, at which case it sanitizes the output leaving $a^{\text{gcd}( j,\ k)}$ written on the tape. $\checkmark$


\section*{Problem 56}

Since $A$ and $B$ are recursive, there are Turing Machines that decide both of them, which I will call $M_A$ and $M_B$. I will construct a TM $N$ that decides $A \cup B$.

The first thing that $N$ does is to replace each symbol to the right of the left end-marker with a pair of the same symbols, vertically stacked. The top symbols will serve as a tape for $M_A$, and the bottom a tape for $M_B$.

The machine then simulates $M_A$ on the top tape symbols, moving the head as $M_A$ dictates and modifying the symbols as directed. If $M_B$ ever writes a symbol to a space that hasn't been divided up as in paragraph two, we do it as soon as $M_B$ writes the symbol. Since $A$ is recursive, $M_A$ is guaranteed to halt; once it does, we do one of two things depending on the halt state If $M_A$ accepted the string $x$, then $x \in A$ implying $x \in A \cup B$, so $N$ immediately accepts. 

If $M_A$ rejects, we repeat the previous process, this time with $M_B$ and the bottom side of the tape, which has not been touched since we first made the two tapes. Again, if $M_B$ accepts, so does $N$. However, this time if $M_B$ rejects, $N$ also rejects.

By simulating each machine separately, we can check if our string is in $A$ or in $B$; if it either, then we accept, else we reject. Therefore $N$ accepts an input iff the input is in $A \cup B$. $\checkmark$




\end{document}