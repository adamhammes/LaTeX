\documentclass[11pt]{article}

\usepackage{enumerate}
\usepackage{amsmath}
\usepackage{amssymb}
\usepackage{parskip}

\let\iff\leftrightarrow
\let\imp\rightarrow

\begin{document}

\title{Com S 331 Homework 10}
\author{Adam Hammes $\bullet$ hammesa@iastate.edu}
\date{November 19, 2014}
\maketitle

\section*{Problem 73}

By Theorem 5 in lecture, there exists an infinite number of strings $x \in \{0,1\}^n$ such that $C(x) < n$.
Note that $\forall n \in \mathbb{N}$, $T(n) > n$.
Since there are an infinite number of strings whose Kolmogorov complexity is less than their length, there is an infinite number of strings $x \in \{0,1\}^*$ such that $C(x) < T(|x|)$. $\checkmark$


\section*{Problem 74}

Consider a Turing maching M that simply stores the result of the string pairing function and outputs it straight from memory.
$M$ needs to store the two strings; it also needs to store a number of zeroes equal to the length of one of the strings.
Without loss of generality, we can store the length of the shorter string.

Instead of storing the strings directly, we can store the Turing machines that output the strings.
The size of $M$ therefore is equal to $C(x) + C(y)$ + min$\{C(x), C(y)\} + c$, where $c$ is a constant unique to the machine.
Since $M$ can store any string pairing result, it acts as an upper bound to the string pairing function. $\checkmark$


\section*{Problem 75}
 
Suppose for contradiction that max$\{C(x),C(y)\} -c $ is not a lower bound on the string pairing function.
Then there exists $x, y \in \{0,1\}^*$ such that $C( \langle x, y \rangle ) < $ max $\{C(x), C(y)\} -c$.

Let $C(x) = a$, and $\langle x, y \rangle = z$. Because $C(x) > C(y)$, $C(z) = C(x) -c$ where $c \in \mathbb{N}$ is a constant.
Because a Turing machine can output $x$ given $z$, $C(x) < C(z)$, implying $C(x) < C(x) -c$, a contradiction.
Therefore max$\{C(x),C(y)\} -c $ is a lower bound on the complexity of the string pairing function. $\checkmark$


\section*{Problem 76}

Consider a language $A$, and let the Turing machine $M$ decide $A$.
By listing the elements of $A$ in the standard enumeration of $\{0,1\}^*$, you can represent strings in $A$ by a natural number $n$ indicating thier position in the sequence.

Given $n$, a Turing machine $M'$ can produce the original string by iterating through the standard enumeration of $\{0,1\}^*$ until $M$ has accepted $n$ elements.
Therefore the Kolmogorov complexity of $a \in A$ is equal to the number of elements in $A$ the same length or shorter, as that is what determines the size of $n$.
The complexity of $a$ is at most the optimality constant $c$ of $M'$ plus the $C$(maximum value of $n$), so
\[C(a \in A) \leq c + \text{log} \mid A \cap \{0,1\}^{\leq |x|}\ \checkmark \]


\section*{Problem 77}

A the last half of a palindrome is the same as the first half, a fact we can exploit with the following TM $M'$:

\begin{itemize}
	\item store $x'$, the first half of $x \in $ PAL
	\item store $b$, a bit indicating whether $|x|$ is even
	\item if $|x|$ is even, output $x' \cdot rev( x' )$; if $|x|$ is odd, don't put the last bit of $x'$ in $rev( x' )$
\end{itemize}

The size of $M'$ depends only on $x'$ and an optimality constant $c$ (which includes $b$). Therefore
	\[C(x \in \text{PAL}) \leq \dfrac{|x|}{2} + c \ \checkmark\]

\section*{Problem 78}

Let $M$ be as follows:

\begin{itemize}
	\item store $x$
	\item store a list $s$ of natural numbers indicating the bits in $x$ to flip to get $y$
	\item output $y$ by flipping the bits of $x$ indicated by $s$
\end{itemize}

Note that $C$(y) is dependent on three things: $C(x)$, $C(s)$, and the optimality constant $c$ of $M$.
No number in $s$ will be greater than $|x|$, so their Kolgomorov complexity will be at most $C$(log $x$).
The size of $s$ is equal to the Hamming distance of $x$ and $y$
Therefore
\begin{align*}
	C(y) & \leq C(x) + H(x, y) \cdot \text{log } x  + c \\
	&\leq C(x) + 2 \cdot H(x, y) \cdot \text{log } x  + c\ \checkmark
\end{align*}


\section*{Problem 79}

Let $A$ be a decidable set and $M$ the TM that decides it.
Define $M'$ as follows:

\begin{itemize}
	\item store $n \in N$
	\item let $x$ represent $\chi_{A,n}$
	\item for $i$ from 1 to $n$:
	\begin{itemize}
		\item let $x_i$ be the i$^{\text{th}}$ bit of $x$
		\item if $M$ accepts $i$, $x_i$ = 1; else $x_i$ = 0
	\end{itemize}
	\item output $x$
\end{itemize}

The size of $M'$ depends only on the complexity of $n$ and an optimality constant $c_{M'}$; therefore $\forall n \in \mathbb{N}$
	\[C(\chi_{A,n}) \leq \text{log }n +c\]
by Observation 8. 


\section*{Problem 80}

Not attempted.

\end{document}