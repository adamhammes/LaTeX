\documentclass[11pt]{article}

\usepackage{enumerate}
\usepackage{amsmath}
\usepackage{amssymb}
\usepackage{parskip}

\let\iff\leftrightarrow
\let\imp\rightarrow

\begin{document}

\title{Com S 331 Homework 10}
\author{Adam Hammes $\bullet$ hammesa@iastate.edu}
\date{November 19, 2014}
\maketitle

\section*{Problem 73}

By Theorem 5 in lecture, there exists an infinite number of strings $x \in \{0,1\}^n$ such that $C(x) < n$.
Note that $\forall n \in \mathbb{N}$, $T(n) > n$.
Since there are an infinite number of strings whose Kolmogorov complexity is less than their length, there is an infinite number of strings $x \in \{0,1\}^*$ such that $C(x) < T(|x|)$. $\checkmark$


\section*{Problem 74}

Consider a Turing maching M that simply stores the result of the string pairing function and outputs it straight from memory.
$M$ needs to store the two strings; it also needs to store a number of zeroes equal to the length of one of the strings.
Without loss of generality, we can store the length of the shorter string.

Instead of storing the strings directly, we can store the Turing machines that output the strings.
The size of $M$ therefore is equal to $C(x) + C(y)$ + min$\{C(x), C(y)\} + c$, where $c$ is a constant unique to the machine.
Since $M$ can store any string pairing result, it acts as an upper bound to the string pairing function. $\checkmark$


\section*{Problem 75}
 
Suppose for contradiction that max$\{C(x),C(y)\} -c $ is not a lower bound on the string pairing function.
Then there exists $x, y \in \{0,1\}^*$ such that $C( \langle x, y \rangle ) < $ max $\{C(x), C(y)\} -c$.

Let $C(x) = a$, and $\langle x, y \rangle = z$. Because $C(x) > C(y)$, $C(z) = C(x) -c$ where $c \in \mathbb{N}$ is a constant.
Because a Turing machine can output $x$ given $z$, $C(x) < C(z)$, implying $C(x) < C(x) -c$, a contradiction.
Therefore max$\{C(x),C(y)\} -c $ is a lower bound on the complexity of the string pairing function. $\checkmark$




\end{document}