\documentclass[11pt]{article}
\usepackage{enumerate}
\usepackage{amsfonts}
\usepackage{amsmath}
\usepackage{mathabx}
\usepackage{graphicx}
\usepackage{adjustbox}
\usepackage{parskip}


\begin{document}

\title{Cpr E Assignment 6}
\author{Adam Hammes $\bullet$ 887020134 $\bullet$ Section M}
\maketitle

\section*{Problem 1}
\begin{enumerate}[(a)]
	\item 8 bits
	\item 12 bits
	\item 21 bits
\end{enumerate}

\section*{Problem 2}
\begin{enumerate}[(a)]
	\item $2^6 -1 = 63$
	\item $2^{10} -1 = 1,023$
	\item $2^{16}-1 = 65,535$
\end{enumerate}

\section*{Problem 3}
\begin{enumerate}[(a)]
	\item 13
	\item 21
	\item 78
	\item 256
\end{enumerate}

\section*{Problem 4}
\begin{enumerate}[(a)]
	\item 1001
	\item 1110
	\item 111
	\item 11001000
	\item 111111111
\end{enumerate}

\section*{Problem 5}
Let each finger represent a binary number, with an extend finger representing a "1" bit and a non-extended finger representing "0". Since the jogger can't run a negative number of laps, we can let her fingers represent unsigned integers. With 10 digits (pun intended) the maximum number of laps that can be conveniently counted is $2^{10} -1 = 1,023$.

\section*{Problem 6}
\begin{enumerate}[(a)]
	\item ceiling( $\log _3 (255+1)$ ) = 6
	\item ceiling( $\log _3 (4,095+1)$ ) = 8
	\item ceiling( $\log _3 (1,234,567)$ ) = 13
\end{enumerate}

\section*{Problem 7}
\begin{enumerate}[(a)]
	\item D
	\item 15
	\item 4E
	\item 100
\end{enumerate}

\section*{Problem 8}
\begin{enumerate}[(a)]
	\item 11000011
	\item 11111110010
	\item 11111010110011101101
\end{enumerate}


\end{document}