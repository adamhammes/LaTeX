\documentclass[11pt]{article}
\usepackage{enumerate}
\usepackage{amsfonts}
\usepackage{amsmath}
\usepackage{mathabx}
\usepackage{graphicx}
\usepackage{adjustbox}
\usepackage{parskip}


\begin{document}

\title{Cpr E Assignment 6}
\author{Adam Hammes $\bullet$ 887020134 $\bullet$ Section M}
\maketitle

\section*{Problem 1}
\begin{enumerate}[(a)]
	\item ceiling( $\log _2 (255+1)$ ) = 8 bits
	\item ceiling( $\log _2 (4,095+1)$ ) = 12 bits
	\item ceiling( $\log _2 (1,234,567-1)$ ) = 21 bits
\end{enumerate}

\section*{Problem 2}
\begin{enumerate}[(a)]
	\item $2^6 -1 = 63$
	\item $2^{10} -1 = 1,023$
	\item $2^{16}-1 = 65,535$
\end{enumerate}

\section*{Problem 3}
\begin{enumerate}[(a)]
	\item $2^3+ 2^2+2^0=13$
	\item $2^4+2^2+2^0 =21$
	\item $2^6 + 2^3 + 2^2 +2^1 =78$
	\item $2^8 =256$
\end{enumerate}

\section*{Problem 4}
\begin{enumerate}[(a)]
	\item 1001
	\item 1110
	\item 111
	\item 11001000
	\item 111111111
\end{enumerate}

\section*{Problem 5}
Let each finger represent a binary number, with an extend finger representing a "1" bit and a non-extended finger representing "0". Since the jogger can't run a negative number of laps, we can let her fingers represent unsigned integers. With 10 digits (pun intended) the maximum number of laps that can be conveniently counted is $2^{10} -1 = 1,023$.

\section*{Problem 6}
\begin{enumerate}[(a)]
	\item ceiling( $\log _3 (255+1)$ ) = 6
	\item ceiling( $\log _3 (4,095+1)$ ) = 8
	\item ceiling( $\log _3 (1,234,567+1)$ ) = 13
\end{enumerate}

\section*{Problem 7}
For this problem and problem 8, I used the 4-bit substitution technique taught in recitation, i.e. $0000_2 = 0_{16}$, $0001_2 = 1_{16}$, etc.
\begin{enumerate}[(a)]
	\item D
	\item 15
	\item 4E
	\item 100
\end{enumerate}

\section*{Problem 8}
\begin{enumerate}[(a)]
	\item 11000011
	\item 11111110010
	\item 11111010110011101101
\end{enumerate}

\section*{Problem 9}
\begin{enumerate}[(a)]
	\item $123 = 7 \times 16 + 11 =$ 7B
	\item $210 = 13 \times 16 + 2 =$ D2
	\item $1023 = 1024 -1 = 400-1 = $ 3FF
\end{enumerate}

\section*{Problem 10}
\begin{enumerate}[(a)]
	\item 4 * 16 + 15 * 1 = 79
	\item 10 * 16 + 1 * 1 = 161
	\item 3* 256 + 13 * 16 + 8 * 1 = 984
\end{enumerate}

\section*{Problem 11}
Let $b$ be the unknown base.
\begin{align*}
5(8)^2 - 5b(8) + b^2 + 2b + 5 &= 0 \\
320 - 40b+b^2+2b+5 &= 0 \\
b^2 -38b+ 325 &= 0 \\	
b &= 13, 25 \\
\end{align*}

To find which solution is correct we must use the other solution to the original equation, 5.
\begin{align*}
5(5)^2 - 5(13)(5) +13^2+2(13)+5 &\stackrel{?}{=} 0\\
0 &= 0\ \checkmark
\end{align*}

Therefore $b$ = 13. By making some very unfounded assumptions about Martian anatomy, we can also conclude that Martians had 13 fingers.

\section*{Problem 12}
\begin{enumerate}[(a)]
	\item -21
	\item 10
	\item -11
\end{enumerate}

\section*{Problem 13}
\begin{enumerate}[(a)]
	\item 1111
	\item 0100
	\item 1001
\end{enumerate}

\section*{Problem 14}
\begin{enumerate}[(a)]
	\item 1010
	\item 1111
	\item 0110
\end{enumerate}

\section*{Problem 15}
\begin{enumerate}[(a)]
\item
	$1111_2+0001_2 =0000_2$; no overflow
\item
	$1000_2+1110_2 = 0110_2$; overflow occurs
\item
	$1111_2 - 0001_2 = 1111_2 + 1111_2 = 1110_2$; overflow occurs
\item
	$1000_2-1101_2 = 1000_2 + 0011_2 = 0101_2$; no overflow

\end{enumerate}

\end{document}