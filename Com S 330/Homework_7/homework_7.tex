\documentclass[11pt]{article}
\usepackage{enumerate}
\usepackage{amsfonts}
\usepackage{amsmath}
\usepackage{mathabx}

\begin{document}

\let\iff\leftrightarrow

\title{Com Sci 330 Assignment 7}
\author{Adam Hammes}
\maketitle

\section*{Problem 1}
	\begin{enumerate}[(a)]

	\item \begin{description}
			
		\item[Base Case:] $n=1$
		
		$6 \times 1 = 6$. $6 \in S$ by the base definition of S.
	
		\item[Inductive Step:]
		Assume $6k \in S$. Prove $6(k+1) \in S$.
		
		$6(k + 1) = 6k + 6$. $6k \in S $ by the Inductive Hypothesis; $6 \in S
		$ by the base definition of $S$. Therefore $6(k+1) \in S$ by the
		inductive step of the definition of $S$.
	

	\end{description}
	
	Therefore $A \subseteq S$
	
	\item \begin{description}
		\item[Basis:] $6 \in S$. $6 \times 1 = 6$, $1 \in \mathbb{Z}^+$, 					therefore $6 \in A$
		
		\item[Inductive Step:] Consider $x, y \in S$. Assume $x, y \in A$ prove
		$x + y \in A $
		
		\begin{align*}
		x, y &= 6a, 6b \text{ where } a, b \in \mathbb{Z} && \text{Inductive 
		Hypothesis} \\		
		x + y &= 6a + 6b \\
		&= 6( a + b )
		\end{align*}
			 
	\end{description}
	
	$a+b \in \mathbb{Z}$, so $6(a+b) \in A$. $6(a+b) = x + y$; therefore $x+y 
	\in A $, concluding the inductive step.
	\end{enumerate}

\section*{Problem 2}
	\begin{enumerate}[(a)]
		\item			
			(1,3), (3,1) \\
			(2, 6), (4,4), (6,2)	\\
			(3,9), (5, 7), (7, 5), (9, 3) \\
			(4, 12), (6, 10) (8, 8), (10, 6), (12, 4)\\
			(5, 15), (7, 13), (9, 11), (11, 9), (13, 7), (15, 5)
			
		\item \begin{description}
			
		\item[Base Case:] Zero applications of the inductive step of the 
		definition.\\
		
		Zero applications of the inductive step gives us the base case, $(0,\ 0)
		$. $0+0 = 0$; $4 \mid 0$, proving the base case.
		
		\item[Inductive Step:] Assume that with $k$ or fewer applications of the
		recursive definition of $S$ we have $(x,\ y) \in S$, $4 \mid x + y$ . 
		Let $(x',\ y')$ denote the result of applying the recursive step one 
		more time. Prove $4\mid x' + y'$.
		
		
		There are two cases. Either $(x',\ y') = (x + 1,\ y+3)$ or $(x',\ y') = 
		(x + 3,\ y+1)$. In both cases, $x' + y' = x + y + 4$. By the inductive
		hypothesis, $x + y = 4a, a \in \mathbb{Z}$; therefore $x' + y'$ can be
		written as $4(a + 1)$. Since $a+1 \in \mathbb{Z}$, $4 \mid x' + y'$,
		concluding the inductive step.
		\end{description}
		
		\item \begin{description}
		
		\item[Base Case:] $(0,\ 0)$.	$0 + 0 = 0$ and $4 \mid 0$, proving the
		base case.
		
		\item[Inductive Step:] Assume $(x,\ y) \in S$ and $4 \mid x+y$. Let
		$(x',\ y')$ denote the result of applying the recursive step to $(x,\ y)
		$. Prove $4 \mid x' + y'$.
		
		There are two cases. Either $(x',\ y') = (x + 1,\ y+3)$ or $(x',\ y') = 
		(x + 3,\ y+1)$. In both cases, $x' + y' = x + y + 4$. By the inductive
		hypothesis, $x + y = 4a, a \in \mathbb{Z}$; therefore $x' + y'$ can be
		written as $4(a + 1)$. Since $a+1 \in \mathbb{Z}$, $4 \mid x' + y'$,
		concluding the inductive step.
		
		\end{description}
	
		\item
		
		Let $a$ be the sum of $(x,\ y) \in S$. With each application of the
		recursive step $a$ increases. Since by the second application of the
		recursive step $a = 8$. Since $ 2 + 2 = 4 < 8 $ and $(2,\ 2)$ is not in 
		the base case or produced by the first application of the 
		recursive step,  $(2,\ 2) \notin S$.
		
		
		To change the recursive definition to include all positive integers
		divisible by 4, add the following: if $(x,\ y) \in S$, $ (x+2,\ y+2) \in 
		S$, $(x+4,\ y) \in S$, and $(x,\ y+4) \in S$.	
	
	\end{enumerate}





	
	
\section*{Problem 3}
	\begin{description}
		\item[Base Case:] Consider the full binary tree consisting of a single
		vertex. This tree has one leaf and zero internal nodes. Therefore the 
		number of leaves is one more than the number of internal nodes, proving
		the base case.
		
		\item[Inductive Step:] Let $T_1$ and $T_2$ be left and right subtrees
		of $T$. By the inductive hypothesis, for both $T_1$ and $T_2$ the number
		of leaves is one greater than the number of internal nodes. $T$ combines
		$T_1$ and $T_2$ with an additional internal node (the root node);
		therefore the number of leaves is $2-1 = 1$ greater than the number of
		internal nodes, concluding the inductive step.
	
	\end{description}
	
	
	Taken together, the base case and inductive steps prove the proposition that 
	for any full binary tree, the number of leaves is one greater than the
	number of internal nodes.

\section*{Problem 4}
	\begin{description}
	
	\item[Base Definition:] $\epsilon,\ a,\  b,\ c \in S$.
	
	\item[Inductive Definition:] If $s \in S$,
	\begin{align*}	
		\text{a}s\text{a} \in S \\
		\text{b}s\text{b} \in S \\
		\text{c}s\text{c} \in S		
	\end{align*}
	
	\end{description}

\section*{Problem 5}
	\begin{enumerate}[(a)]
	
	\item
	\begin{description}
		\item[Base Case:] $ 1 \in S $
		\item[Inductive Cases:] $n \in S \rightarrow 2n, 3n, 5n \in S$
	\end{description}
	
	\item
	\begin{description}
		\item[Base Case:] $ 1 \in S$
		\item[Inductive Cases:] $n \in S \rightarrow 5n, 15n, 18n \in S$
		
	\end{description}

	\end{enumerate}
	
\section*{Problem 6}
	\begin{enumerate}[(a)]
		\item \begin{description}
		
		\item[Base Case:] $(0,\ 0) \in S$
		
		\item[Inductive Step:] If $(x,\ y) \in S$, then
		
		\begin{enumerate}[i]
			\item $(x + 3,\ y) \in S$
			\item $(x-3,\ y) \in S$
			\item $(x,\ y + 3) \in S$
			\item $(x,\ y-3) \in S$
			\item $(x -1,\ y-2) \in S$
			\item $(x -2,\ y-1) \in S$
			\item $(x +1,\ y+2) \in S$
			\item $(x +2,\ y+1) \in S$
		\end{enumerate}		
		\end{description}	
	
		\item  I prove $L' \subseteq L$ using structural induction.
		
		\begin{description}
		\item[Base Case:] $(0,\ 0)$
		
		$0 + 0 = 0$; $0\mod 3 = 0$, proving the base case.
		
		\item[Inductive Step:] Consider $(x,\ y) \in S$, with $x + y \mod 3 = 0$
		. Let $(x', y')$ denote the application of the recursive step of the
		definition of $S$ on $(x',\ y')$. We prove that for all possible values 
		of $(x',\ y')$ that $3 \mod x' + y' = 0$.
		

		\textit{Lemma 1}: If $3 \mid a$ and $3 \mid b$, $a + b \mod 3 = 0$	. 
		Proof: Since
		$3 \mid a, b$ then $a + b$ can be written as $3( m + n)$, $m,n \in 
		\mathbb{Z}$. Since $m+n$ is also $\in \mathbb{Z}$, $3 \mid a+ b$ and
		$a+b \mod 3$ = 0.
		
		
		In all cases, $x' + y' = x + y \pm 3$. Since $3 \mid x+y$
		and $3 \mid 3, -3$, by Lemma 1 $x' + y' \mod 3 = 0$, concluding the
		inductive step.
		
		\end{description}
		
		
		\item Not attempted.
		
	
	
	\end{enumerate}



\section*{Extra Credit}


Consider $ (3,\ 3) $. $3 + 3 =6$, and $6 \mod 3 = 0$, so $(3,\ 3) \in S$. However, $(3,\ 3)$ can be constructed in multiple ways, two of which I'll enumerate:
	\begin{align*}
		(0,\ 0) \rightarrow 	(0,\ 3) \rightarrow (3,\ 3)\\
		(0,\ 0) \rightarrow 	(1,\ 2) \rightarrow (3,\ 3)
	\end{align*}

The fact that $(3,\ 3)$ can be formed through 2 ways proves that our definition is ambiguous. This definition is not ambiguous:

\begin{description}

	\item[Base Case:] $(0,\ 0)\in S$

	\item[Recursive Definition:] If $(x,\ y) \in S$, then

	\begin{enumerate}[{}]
		\item $(x + 3,\ y) \in S \iff x \ge 0 \bigwedge 3 \mid x,\ y$
		\item $(x - 3,\ y) \in S \iff x \le 0 \bigwedge 3 \mid x,\ y$
		\item $(x,\ y + 3) \in S \iff y \ge 0 \bigwedge 3 \mid x,\ y$
		\item $(x,\ y - 3) \in S \iff y \le 0 \bigwedge 3 \mid x,\ y$ 
		\item $(x -1,\ y-2) \in S \iff x \le 0 \bigwedge x \mod 3 \neq 1 $
		\item $(x -2,\ y-1) \in S \iff x \le 0 \bigwedge x \mod 3 \neq 2 $ 
		\item $(x +1,\ y+2) \in S \iff x \ge 0 \bigwedge x \mod 3 \neq 2 $
		\item $(x +2,\ y+1) \in S \iff x \ge 0 \bigwedge x \mod 3 \neq 1 $
	\end{enumerate}		


\end{description}









\end{document}