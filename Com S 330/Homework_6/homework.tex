\documentclass[11pt]{article}
\usepackage{enumerate}
\usepackage{amsfonts}
\usepackage{amsmath}
\usepackage{mathabx}

\begin{document}

\title{Com Sci 330 Assignment 6}
\author{Adam Hammes}
\maketitle

\section*{Problem 1}
	\begin{enumerate}[(a)]
		\item $a_0 = 2$\\
			  $a_{n+1} = a_n + 4$
	\end{enumerate}
	\begin{enumerate}[(d)]
		\item $a_0 = 1$ \\
			  $a_{n+1} = (\sqrt{a_n} + 1)^2$
	\end{enumerate}

\section*{Problem 2}
	\begin{description}

	\item[Base case:] $n =1$
	\begin{align*}
		f_{2}f_0 - f_1^2 &= (-1)^n\\
		1*0 -1^2 &= (-1)^1 \\
		-1 &= -1 && \text{True}
	\end{align*}
	Therefore we have proved the base case.
	
	\item[Inductive Step:] Assume $f_{n+1}f_{n-1}-f_n^2 = (-1)^n$. Prove $f_(n+2)f_n -f(n+1)^2 = (-1)^{n+1}$.
	
	\begin{align*}
		f_{n+2}f_{n+1}-f_{n+1}^2 &= (-1)^{n+1} \\
		f_{n+2}f_{n+1}-f_{n+1}^2 &= (-1)^n*-1 \\
		f_n^2 -f_{n-1}f_{n+1} &= (-1)^n\\
		f_n^2-(f_{n+1}-f_n)f_{n+1} &= (-1)^{n+1}\\
		f_n^2-f_{n+1}^2 + f_nf_{n+1} &= (-1)^{n+1}\\
		f_n(f_n+f_{n+1})-f_{n+1}^2 &= (-1)^{n+1}\\
		f_nf_{n+1}-f_{n+1}^2 &= (-1)^{n+1} && \text{same as what we assumed}
	\end{align*}
	Therefore we have completed the inductive step.
	\end{description}	
	Taken together, the base case and inductive step prove $f_{n+1}f_{n-1} - f_n^2 = (-1)^n, n \in \mathbb{Z}^+$
	

\section*{Problem 3}
	\begin{description}
	\item[Base Case:] $n=1$
	\begin{align*}
	f_0 -f_1 + f_2 &= f_{2-1} -1 \\
	0 -1 + 1 & = 1-1 \\
	0 &= 0 && \text{True}
	\end{align*}		
	Therefore we have proved the base case.
	
	\item[Inductive Step:]
	\begin{description}
	\item{} Assume: $ f_0-f_1+f_2-\ldots - f_{2n-1} + f_{2n} = f_{2n-1} -1$
	\item{} Prove:  $ f_0-f_1+f_2-\ldots - f_{2n-1} + f_{2n} -f_{2n+1}+f_{2n+2} = f_{2n+1} -1$
	
	\end{description}
	
	\begin{align*}
	f_0-f_1+f_n-\ldots-f_{2n-1} + f_{2n} -f_{2n+1}+f_{2n+2} &= f_{2n+1} -1 \\
	f_{2n-1} -1 -f_{2n+1}+f_{2n+2} &= f_{2n+1} -1 && \text{Inductive Hypothesis}\\
	f_{2n-1}-f_{2n+1}+f_{2n+2} &= f_{2n+1}\\
	f_{2n-1}-f_{2n+1}+f_{2n}+f_{2n+1} &= f_{2n+1}\\
	f_{2n-1}f_{2n} &= f_{2n+1}\\
	f_{2n+1} &= f_{2n+1} && \text{True}
	\end{align*}
	Therefore we have completed the inductive step.
	\end{description}
	
	Taken together, the base case and inductive step prove $f_0-f_1+f_2-\ldots-f_{2n-1} + f_{2n} = f_{2n-1} -1, n\in\mathbb{Z}^+$.
	
	



\section*{Problem 4}
Prove: Let $L(a)$ be the proposition that a length of the periphery of a structure of $a$ squares is even. $\forall n \ge 0, L(n)$.

	\begin{description}
	
		\item[Base case:] Prove $L(1)$.
		
		
			The length of the periphery is equal to the square's perimeter, 
			which is four. Four is divisible by two, so $L(1)$.
	
		\item[Inductive case:] Assume $L(k)$. Prove $L(k+1)$.
		
		
			Every edge that does not border the added square remains in
			the periphery; it follows that we need only consider only 
			the edges touched by the new square and the edges of the 
			square itself when recalculating the perimeter. 
			
			By the inductive hypothesis, the perimeter can be written as $2m, 
			m \in \mathbb{Z}$. If the net change in perimeter is even, then it 
			can be 
			written as $2m + 2n, n \in \mathbb{Z}$, and the 
			resulting shape has an even-length perimeter. Looking at 
			the possible	 placement of the new square, there are four 
			cases.
		
			\begin{enumerate}[{Case } 1:]
				\item The square shares one edge with the shape. The shape 
				loses one unit of periphery from the shared edge and gains 
				three units from the exposed edges of the new square, 
				resulting in a net change of $3-1 = 2$ units. Two is even, 
				therefore the resulting shape has an even-length periphery.
				
				\item The square shares two edges with the shape. The shape 
				loses two units of periphery from the shared edges and gains 
				two units from the exposed edges of the new square, resulting 
				in a net change of $2-2 = 0$ units. Zero is even, therefore 
				the resulting shape has an even-length periphery.
				
				\item The square shares three edges with the shape. The shape 
				loses three units of periphery from the shared edges and gains 
				one unit from the exposed edge of the new square, resulting in 
				a net change of $1-3 = -2$ units. Negative two is even, 
				therefore the resulting shape has an even-length periphery.
				
				\item The square shares all four edges with the shape. The 
				shape loses 4 units of periphery from the the shared edges and 
				gains no units back from the square, resulting in a net loss 
				of four units. Negative four is even, therefore the resulting 
				shape has an even length periphery.
			
			\end{enumerate}
	\end{description}					
				Since one square has an even length periphery, and all 
				possible additions of squares result in an even length 
				periphery, all possible arrangement of squares results in an 
				even length periphery.
			
			
		
\section*{Problem 5}
	\begin{enumerate}[(a)]
		\item 3, 5, 6, 7
		\item 4, 5, 7, 8
	\end{enumerate}


\section*{Problem 6}
	Strengthened inductive hypothesis: after $k$ steps, the state machine is in state $k \bmod 4$.
	
	\begin{description}
	\item[Base Case:] $n=0$\\
	If no steps are taken, then the machine is in state $0$. $0 \bmod 4=0$, so we have proved the base case.

	\item[Inductive Step:] Assume that after $n$ steps, the machine is in state $n \bmod 4$. Prove that in $n+1$ steps, the machine is in state $(n+1)\bmod 4$.
	
	By corollary 2 of section 1 of chapter 4 in Rosen's textbook, $(n+a) \bmod m = ((a\bmod m) + (b\bmod m)) \bmod m$. Since $1\bmod4 = 1$ and $ m, n\in\mathbb{Z}$, $(n\bmod 4) + 1 = (n+1)\bmod 4$, proving the inductive step.

	\end{description}
By the preceding proof, if after $n$ steps we are in state $k$, then after $n+1$ we are in state $0$ if and only if $k=3$.
	
	




\section*{Problem 7}

	\begin{description}
		\item[States:] $\{x, y \mid x, y \in \mathbb{Z}\}$
		\item[Start states:] $(0, 0)$
		\item[Transitions:] 
			$\{(x, y) \rightarrow (x+1, y+2) \mid x, y \in \mathbb{Z} \} \cup \{(x, y) \rightarrow (x+3, y) \mid x, y \in \mathbb{Z} \} \cup \{(x, y) \rightarrow (x-2, y-1) \mid x, y\in \mathbb{Z} \}$
			
		\item[Invariant:] For state $R= (m, n)$ reachable from $(0, 0)$, $3 
		\mid m+n$
		 
		\item[Proof:] The base case is true because $0 + 0 = 0$, and zero is 
		divisible by 3.	We prove that for any state $S = (a, b)$, if $3 \mid a 
		+ b$, all states leading from $S$ this property also holds.

	Let $Q$ be a state such that	$ R \rightarrow Q $. There are three possibilities for the value of $Q$, one for each possible transition.
	
	\begin{description}
		\item{Case 1:} $Q = (a+1,\, b+2)$.
\begin{align*}
    (a+ 1) + (b+2) &= a + 1 + b + 2 \\
    &= a + b + 3\\
    &= 3l + 3,\,l\in \mathbb{Z}  && \text{Inductive Hypothesis}\\
    &= 3(l + 1),\, l + 1\in\mathbb{Z}
\end{align*}
	Therefore the Invariant Principle holds.
	
		\item{Case 2:} $Q = (a+3,\, y)$
		\begin{align*}
		(a+3) + (b) &= a + b + 3 \\
		&= 3(l + 1),\, l+1\in\mathbb{Z} &&\text{From Case 1}	
		\end{align*}
	Therefore the Invariant Principle holds.
	
		\item{Case 3:} $Q = (a-2,\,b-1)$
		\begin{align*}
		(a-2) + (b-1+ &= a - 2 + b -1 \\
		&= a + b -3\\
		&= 3l -3, l\in\mathbb{Z} && \text{Inductive Hypothesis}\\
		&= 3(l + 1), l\in\mathbb{Z}
		\end{align*}
	Therefore the Invariant Principle holds for all states reachable from $Q$, and we have proved the Inductive Step. This, along with the base case, allow us to conclude that for any state $A = (c,\,d)$, if $A$ is reachable then $3 \mid (c + d)$. Since $1 + 1 = 2$ and $ 3 \notdivides 2$, the robot will never get to $(1, 1)$.
		
	
	
	\end{description}
	
	\end{description}
	



\end{document}