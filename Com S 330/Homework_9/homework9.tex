\documentclass[11pt]{article}
\usepackage{enumerate}
\usepackage{amsfonts}
\usepackage{amsmath}
\usepackage{mathabx}

\begin{document}

\let\iff\leftrightarrow

\title{Com Sci 330 Assignment 9}
\author{Adam Hammes, hammesa@iastaste.edu}
\maketitle

\section*{Problem 1}
	\begin{enumerate}[(a)]
	\item 
		$f(n) = \begin{cases}
			\dfrac{-5n}{2}, &\text{if } n \text{ is even}\\[2ex]
			\dfrac{5(n+1)}{2}, &\text{if } n \text{ is odd}
			\end{cases}$
			
		where $n \in \mathbb{N}$.
		
	\item 
	
		$f(n) = \begin{cases}
		\dfrac{-5n+ 2\ \lfloor{ \dfrac{n}{14}}\rfloor}{2}, &\text{if } n \text{ is even}\\[2ex]
		
		\dfrac{5(n+1+2\ \lfloor{ \dfrac{n+1}{14} } \rfloor\ )}{2}      &\text{if } n \text{ is odd}
		\end{cases}$
		
		where $n \in \mathbb{N}$.
	
	\item
		$f(n) = \begin{cases}

			\left(-\dfrac{n}{4},\ 0\right) &\text{if } n \mod 4 = 0\\[2ex]
			\left(-\dfrac{n+3}{4},\ 1\right) &\text{if } n \mod 4 = 1\\[2ex] 
			\left(\dfrac{n+2}{4},\ 0\right) &\text{if } n \mod 4 = 2\\[2ex]
			\left(\dfrac{n+1}{4},\ 1\right) &\text{if } n \mod 4 = 3\\
			\end{cases}$

		where $n \in \mathbb{N}$.
	\end{enumerate}
	
\section*{Problem 2}
	Let $A$ and $A \subseteq B$. By definition of countability, there is no function $f: \mathbb{N} \rightarrow A$ such that $f$
	is one-to-one. This can be restated as $\forall f\ \exists a \in A\ \forall\ n \in N\ f(n) \neq a$. Furthermore, since $A \subseteq B,
	\ a \in B$, there also exists no one-to-one function from $\mathbb{N}$ to $B$, proving that $B$ is uncountable.
	
\section*{Problem 3}
	Assume for contradiction there exists an enumeration of containing all the functions from $\mathbb{N}$ to $\{0,\ 1,2,\ \ldots\ ,\ 9\}$, 
	denoted as $f_0,\ f_1,\ f_2,\ \ldots\ , f_i$. Let $G$ be defined as the following:\\
	
	
	$G(i) = \begin{cases}
		1 \text{ if } f_i(i) \geq 5 \\
		9 \text{ if } f_i(i) \le 5
	\end{cases}$, $G:\ \mathbb{N} \rightarrow \{0,\ 1,2,\ \ldots\ ,\ 9\}$
	\\[2ex]
	$\forall n \in N,\ G(n) \neq f_n(n)$, so $G$ is not in the enumeration, a contradiction. Therefore the set is uncountable.
	
	
\section*{Problem 4} For both parts, let $S$ be the set described in the problem.


\begin{enumerate}[(a)]
\item
	Countable. Proof:
	\\\\
	Let each number in the set be represented by an ordered pair $(x,\ y)$ where $x$ is the number of ones before the
	decimal point and $y$ the number after. Since $x,\ y$ are non-negative integers, $(x,\ y) \in \mathbb{N} \times
	\mathbb{N}$, which we proved is countable in class; therefore the set is countable. 
	\\\\
	An example enumeration to further prove countability (note that a 0 is the absence of a 1):
	
	$0.0,\ 1.0,\ 0.1,\ 11.0,\ 1.1,\ 0.11,\ 111.0,\ 11.1,\ \ldots$
	
\item
	Uncountable. Proof:
	\\\\
	Let $0$ symbolize $1$ and and $1$ symbolize $9$. Through a combination of $0$, $1$ and a single decimal point
	we can represent any real number with an infinite length binary decimal; therefore the set is one-to-one with $\mathbb{R}$ and by definition
	of cardinality has the same size. Since $\mathbb{R}$ is uncountable (proved in class) and $|S| =  |\mathbb{R}|$, 
	$S$ is uncountable.


\end{enumerate}


\section*{Problem 5}
	\begin{enumerate}[(a)]
	\item
		$A = \mathbb{R},\ B = \mathbb{R}$
		
	\item
		$A = \mathbb{R},\ B = \mathbb{R} - \mathbb{Z}$
	
	
	\item
		$A = \mathbb{R} \times \{0,\ 1\},\ B = \mathbb{R} \times \{0\}$
	
	
	\end{enumerate}	




\section*{Extra Credit:}
	The set $S$ is countable because it can be enumerated: $\{0,\ 1,\ 2,\ \ldots,\ 9,\ /,\ 00,\ 01,\ 02,\ \ldots\}$
	\\\\
	Positive rationals are defined as $\{ a/b\ |\ a,\ b \in \mathbb{Z}^+\}$. $a,\ b$ can be represented by a string
	of decimal digits, which are in the set, and the division symbol in the set also lets us represent the quotient of those two
	numbers. Since there exists a string in $S$ that represent each positive rational, and $S$ is countable,
	positive rationals are also countable.
	
	
\end{document}